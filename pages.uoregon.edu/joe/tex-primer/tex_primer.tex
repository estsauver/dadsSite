% Save file as: TEX_PRIMER.TEX         Source: FILESERV@SHSU.BITNET  
\input fontsize.tex
\input indexit.tex
\twelvepoint
%%%%%%%%%%%%%%%%%%%%%%%%%%%%%%%%%%%%%%%%%%%%%%%%%%%%%%%%%%%%%%%%%%%%%%%%%%%%%%%
\def\leaderfill{\leaders\hbox to 0.4em{\hss.\hss}\hfill}
%%%%%%%%%%%%%%%%%%%%%%%%%%%%%%%%%%%%%%%%%%%%%%%%%%%%%%%%%%%%%%%%%%%%%%%%%%%%%%%
\nopagenumbers 
\count0=-1
\vglue 2.0truein
\centerline{\stnbf USING \TeX\ ON THE VAX}
\bigskip
\centerline{\stnbf TO TYPESET DOCUMENTS:}
\bigskip
\centerline{\stnbf A PRIMER}
\bigskip\bigskip\bigskip
\centerline{\twlrm DECEMBER 1990}
\vfill
\twlrm
\settabs 5 \columns 
\+&&&{Joseph E. St Sauver}\cr
\+&&&{Assistant Professor/Statistical Programmer}\cr
\+&&&{Office of University Computing}\cr
\+&&&{235 Computing Center}\cr
\+&&&{University Of Oregon}\cr
\+&&&{Eugene, OR 97403 U.S.A.}\cr
\medskip
\+&&&{(503) 346-4394 extension 25}\cr
\+&&&{(503) 346-4397 (FAX only)}\cr
\medskip
\+&&&{Internet: \enspace{\twltt joe@oregon.uoregon.edu}}\cr
\+&&&{BITNET: \thinspace{\twltt joe@oregon}}\cr
\eject
%%%%%%%%%%%%%%%%%%%%%%%%%%%%%%%%%%%%%%%%%%%%%%%%%%%%%%%%%%%%%%%%%%%%%%%%%%%%%%%
\headline={\rm\hfill\folio\voffset=2\baselineskip}
\parindent = 0pt
\global\skip\footins=24pt plus 24pt minus 24pt
\def\footnoterule{\twlrm \vfil \vglue 12pt \hrule width 2truein \vglue 12pt}
%%%%%%%%%%%%%%%%%%%%%%%%%%%%%%%%%%%%%%%%%%%%%%%%%%%%%%%%%%%%%%%%%%%%%%%%%%%%%%%
\centerline{\twlbf PREFACE}
\bigskip\bigskip\par\noindent
\leftline{\twlbf System Dependencies}
^^{system dependencies}
\bigskip\par\noindent
This document contains some system dependent features; to use it at
another site, you should obtain a machine readable copy of the \TeX\
source and modify it appropriately for your system. Obtain a copy 
of this document and any of the files mentioned herein via
anonymous FTP from {\twltt ^{DECOY.UOREGON.EDU} (128.223.32.19)} or contact
the author at the address shown on the title page.
\bigskip\bigskip\par\noindent
\leftline{\twlbf ^{Disclaimers}}
\bigskip\par\noindent
This document is provided \underbar{as is}, without warranty of any
kind, either express or implied, respecting the contents of the document,
including but not limited to implied warranties for the document's 
quality, performance, merchantability or fitness for any particular
purpose. Neither the author nor any other party shall be liable to the
user or any other person or entity with respect to any liability, loss,
or damage caused or alleged to be caused directly or indirectly by this
document. Use the information in this document at your own risk.
\bigskip\bigskip\par\noindent
\leftline{\twlbf ^{Trademarks}}
\bigskip\par\noindent
All trademarks are the property of their respective owners. 
\bigskip\bigskip\par\noindent
\leftline{\twlbf ^{Acknowledgments}}
\bigskip\par\noindent
To Paul Steinman, who first told me about \TeX\ while we were both working
for the University of Alaska-Fairbanks Computer Node many years ago: thanks 
for taking the time to tell me about \TeX{}'s magic!
\bigskip\par\noindent
To Donald Knuth and all the others whose talent, hard work, and
generosity have made \TeX\ the world standard it is today: 
thanks for producing such a great piece of software.
\bigskip\bigskip\par\noindent
\leftline{\twlbf ^{Dedication}}
\bigskip\par\noindent
To Jamie Claire or Earl Joseph, as eventually proves appropriate: we were thinking about you and waiting for you anxiously, even before you counted 
for tax purposes! This document is dedicated to you in the hope that you, 
your children, and your children's children will all be able to enjoy 
the full portfolio of freedoms secured by the Bill of Rights, American
bravery, and the grace of God.
\bigskip\bigskip\par\noindent
\leftline{\twlbf ^{Copyright}}
\bigskip\par\noindent
\copyright\ Joseph E St Sauver, November 1990, all rights reserved. 
Permission to reprint for non-profit purposes is granted provided 
this copyright notice is included and authorship is acknowledged. 
\vfill\eject
%%%%%%%%%%%%%%%%%%%%%%%%%%%%%%%%%%%%%%%%%%%%%%%%%%%%%%%%%%%%%%%%%%%%%%%%%%%%%%%
\centerline{\twlbf ^{CONTENTS}}
\bigskip\par\noindent
PREFACE \leaderfill\quad ii
\bigskip\par\noindent
CONTENTS \leaderfill\quad iii
\bigskip\par\noindent
I. INTRODUCTION
\medskip\par\hglue 0.5truein
What is \TeX? \leaderfill\quad 1
\par\hglue 0.5truein
Why Should I Bother to Learn to Use \TeX? \leaderfill\quad 1
\par\hglue 0.5truein
What Can I Expect of the Rest of This Write-Up? \leaderfill\quad 2
\bigskip\par\noindent
II. ENTERING TEXT (OTHER THAN TABLES AND EQUATIONS) IN \TeX
\medskip\par\hglue 0.5truein
Text Which is Entered Normally in \TeX\ \leaderfill\quad 3
\par\hglue 0.5truein
Special Characters in \TeX\ \leaderfill\quad 4
\par\hglue 0.5truein
Structuring The Text You Enter; Making Paragraphs \leaderfill\quad 6
\par\hglue 0.5truein
Comments \leaderfill\quad 6
\par\hglue 0.5truein
Font Size \leaderfill\quad 7
\par\hglue 0.5truein
Font Style \leaderfill\quad 8
\par\hglue 0.5truein
Underlining \leaderfill\quad 9
\par\hglue 0.5truein
Line Spacing (Double-Spacing, Skipping a Single Line, etc.) \leaderfill\quad 9
\par\hglue 0.5truein
Block Quotations \leaderfill\quad 10
\par\hglue 0.5truein
Centering Text (For Headings, etc.) \leaderfill\quad 11
\par\hglue 0.5truein
Footnotes \leaderfill\quad 12
\par\hglue 0.5truein
Headers and Page Numbers \leaderfill\quad 13
\par\hglue 0.5truein
Page Size; Margins \leaderfill\quad 14
\par\hglue 0.5truein
Leaving Space for Insertions; Forcing Page Breaks \leaderfill\quad 14
\par\hglue 0.5truein
Ending Your \TeX\ Document \leaderfill\quad 14
\bigskip\par\noindent
III. TYPESETTING TABULAR MATERIAL
\medskip\par\hglue 0.5truein
Using Tabs \leaderfill\quad 15
\par\hglue 0.5truein
Typesetting Formal Ruled Tables \leaderfill\quad 16
\par\hglue 1.0truein
Beginning to Decode the Table-Building Commands \leaderfill\quad 17
\par\hglue 1.0truein
Table Template \leaderfill\quad 18
\par\hglue 1.0truein
Table Headings \leaderfill\quad 19
\par\hglue 1.0truein
Table Body \leaderfill\quad 20
\bigskip\par\noindent
IV. TYPESETTING EQUATIONS
\medskip\par\hglue 0.5truein
Typesetting Equations is Different From Typesetting Text \leaderfill\quad 21
\par\hglue 0.5truein
Embedded vs. Displayed Equations \leaderfill\quad 21
\par\hglue 0.5truein
Numbering Equations \leaderfill\quad 22
\par\hglue 0.5truein
Aligning and Numbering Multiple Equations \leaderfill\quad 23
\par\hglue 0.5truein
Some Basic Information about Entering Equations \leaderfill\quad 24
\par\hglue 0.5truein
Greek Letters \leaderfill\quad 25
\par\hglue 0.5truein
Script Letters \leaderfill\quad 25
\par\hglue 0.5truein
Common Mathematical Operators \leaderfill\quad 26
\par\hglue 0.5truein
Symbols for Logic and the Algebra of Sets \leaderfill\quad 26
\par\hglue 0.5truein
Subscripts, Superscripts, and Combinations Thereof \leaderfill\quad 27
\par\hglue 0.5truein
Math Accents \leaderfill\quad 27
\par\hglue 0.5truein
Roman Font Mathematical ``Words'' \leaderfill\quad 28
\par\hglue 0.5truein
Limits \leaderfill\quad 29
\par\hglue 0.5truein
Radicals: Square Roots, Cube Roots, etc. \leaderfill\quad 29
\par\hglue 0.5truein
Making Large Fractions \leaderfill\quad 30
\par\hglue 0.5truein
Making Large Grouping Operators \leaderfill\quad 31
\par\hglue 0.5truein
Combination Notation \leaderfill\quad 32
\par\hglue 0.5truein
Matrices \leaderfill\quad 32
\par\hglue 0.5truein
Case Structure \leaderfill\quad 32
\par\hglue 0.5truein
Summations \leaderfill\quad 33
\par\hglue 0.5truein
Integrals \leaderfill\quad 33
\par\hglue 0.5truein
Definitions \leaderfill\quad 34
\bigskip\par\noindent
V. \TeX\ ON THE OREGON VAX 8800
\medskip\par\hglue 0.5truein
The \TeX\ Execution Cycle \leaderfill\quad 35 
\par\hglue 0.5truein
Building Your \TeX\ Document Using An Editor \leaderfill\quad 36
\par\hglue 0.5truein
Defining \TeX\ \leaderfill\quad 36
\par\hglue 0.5truein
Running \TeX\ \leaderfill\quad 37
\par\hglue 0.5truein
Decoding \TeX\ Errors \leaderfill\quad 38
\par\hglue 0.5truein
The Most Common \TeX\ Errors \leaderfill\quad 39
\par\hglue 0.5truein
Converting Your {\twltt .DVI} File Into PostScript \leaderfill\quad 41
\par\hglue 0.5truein
DVIPS Features \leaderfill\quad 42
\par\hglue 0.5truein
Printing PostScript Output on the VAX's Xerox 4045/160 \leaderfill\quad 43
\bigskip\par\noindent
VI. CONCLUSION
\medskip\par\hglue 0.5truein
Where From Here? \leaderfill\quad 44
\par\hglue 0.5truein
What If I Get Stuck? \leaderfill\quad 44
\bigskip\par\noindent
INDEX
\bigskip\par\noindent
APPENDICES:
\medskip\par\hglue 0.5truein
A: Complete Sample Text-Oriented \TeX\ Document
\par\hglue 0.5truein
B: Complete Sample Technical \TeX\ Document
\par\hglue 0.5truein
C: Sample University of Oregon Thesis Pages from the ``Grey Book''
\par\hglue 0.5truein
D: Some Sample Pages From a Survey Typeset in \TeX\ 
\par\hglue 0.5truein
E: Sample Resume Typeset in Plain \TeX\
\par\hglue 0.5truein  
F: Sample \TeX\ Landscape-Mode Overhead Made Using PostScript Fonts
\par\hglue 0.5truein
G: Demonstration of the Incorporation of PostScript Graphics
\vfill\eject
%%%%%%%%%%%%%%%%%%%%%%%%%%%%%%%%%%%%%%%%%%%%%%%%%%%%%%%%%%%%%%%%%%%%%%%%%%%%%%%
\headline={\rm\hfill\folio\voffset=2\baselineskip}
\count0=1
\centerline{\twlbf I. ^{INTRODUCTION}}
\bigskip\par\noindent
\centerline{\twlbf ^{What is \TeX{}?}}
\bigskip\par\noindent
\TeX{}\footnote{$^{1}$}{Say ``tecchhh'' (as in the greek letter chi),
rather than ``tecks.''\par} is a computerized typesetting program created 
by Stanford's Donald 
Knuth\footnote{$^{2}$}{Of {\twlit Fundamental Algorithms, 
Seminumerical Algorithms,} and {\twlit Sorting and Searching} fame.\par}.
\bigskip\par\noindent 
\TeX\ is {\twlit not} the same as what-you-see-is-what-you-get 
(WYSIWYG) desktop publishing systems such as Aldus PageMaker.
\TeX\ is both more and less than those desktop publishing 
packages. For instance, \TeX\ doesn't require a major investment in an 
expensive microcomputer loaded with lots of RAM and supported by a fast 
hard disk and a high-resolution display --- you can prepare \TeX\ documents 
on a simple terminal. On the other hand, \TeX\ doesn't show you 
the final form of your
document as you prepare it --- you need to first build your document and 
then process it with \TeX\ and another conversion utility before you 
actually get to see the fruits of your typesetting efforts. \TeX\ also
doesn't do a great job of handling integration of text
and graphics.
\bigskip\par\noindent
Similarly, \TeX\ is also {\twlit not} the same as a regular word 
processing package such as Microsoft Word or Word Perfect, 
although \TeX\ could be used to produce virtually any document a 
basic word processor could generate. 
\bigskip\par\noindent
The best way to think about \TeX\ is to think of it as it was truly
meant to be used: \TeX\ is a computerized printing ``press'' 
intended for typesetting scientific and technical manuscripts
of virtually any length up to and including entire books.
\bigskip\par\noindent
Because Knuth has allowed \TeX\ to be freely distributed, \TeX\ has been 
ported to machines ranging from PC's to the largest of IBM mainframes. 
Here at the University of Oregon Computing Center, we run the Northlake 
Software\footnote{$^{3}$}{Northlake Software, 812 SW Washington, Suite 
1100, Portland, OR 97205-3215. Telephone: 503-228-3383. Fax: 
503-228-5662. (Formerly Kellerman and Smith.)} 
VAX/VMS port of \TeX\ on OREGON, the academic/research computing VAX.
\bigskip\bigskip\par\noindent
\centerline{\twlbf Why Should I Bother to Learn to Use \TeX?}
\bigskip\par\noindent
\TeX{}'s biggest attraction is that it can beautifully typeset articles or 
entire books, including complicated mathematical equations. 
No longer do you need to prepare manuscript copy, send it to be 
typed (or typeset) by someone who may not have the slightest 
understanding of what your equations mean, and then tediously 
attempt to spot and correct errors and resubmit corrected proofs 
until you finally get copy which is more-or-less correct. With \TeX\ you 
can produce camera-ready copy (including {\twlit all} of your equations)
laid out exactly the way you want it, without many of the costs, delays or
headaches associated with traditional copy preparation processes.
\bigskip\par\noindent
If you'd like tangible proof that \TeX\ really {\twlit works} for typesetting
complex scientific and technical manuscripts, consider the following
brief list of some outstanding technical books, all of which were all set in 
\TeX\ (or a variant thereof):
^^{Books Typeset in TeX}
\bigskip\par\noindent
{\parindent=0.5truein
\narrower
{\twlit Lisrel 7: A Guide to the Programs and Applications,} 2nd Edition,
Karl G. J\"oreskog and Dag S\"orbom, SPSS Inc, Chicago, Illinois: 1989
(typeset using \TeX{}).
\bigskip\par\noindent
{\twlit Mathematica: A System for Doing Mathematics by Computer,} Stephen 
Wolfram, Addision-Wesley Advanced Book Program, Redwood City, California:
1988 (typeset using \TeX\ and La\TeX{}).
\bigskip\par\noindent
{\twlit Fortran Tools
for VAX/VMS and MS-DOS,} R.K. Jones, John Wiley and Sons,
New York: 1988 (typeset using La\TeX{}).
\bigskip\par\noindent
{\twlit Numerical Recipes: 
The Art of Scientific Computing,} William H. Press,
Cambridge University Press, Cambridge: 1986 (typeset using \TeX{}).
\bigskip\par\noindent
{\twlit The \TeX{}book,} Donald E. Knuth, Addison Wesley, 
Reading, Massachusetts, 1986 (typeset in \TeX{}, of course!).
\bigskip\par\noindent}%
There are many other examples of excellent \TeX{}-created books, but these 
are just a few examples which happened to be handy on my own bookshelf.
\bigskip\par\noindent
Here at the University of Oregon, \TeX\ will probably be used most 
extensively by graduate students preparing theses and dissertations, and
by scientists, mathematicians, and other technical people
preparing manuscripts containing
substantial mathematical notation. \TeX\ is also great for typesetting
professional looking survey instruments and resumes.
\bigskip\bigskip\par\noindent
\centerline{\twlbf What Can I Expect of the Rest of This Write-Up?}
\bigskip\par\noindent
The following sections of this write-up explain how you can create 
your own \TeX\ documents, and how you can convert your \TeX\ document 
into a file suitable for previewing or printing on a laser printer.
\bigskip\par\noindent
This write-up {\twlit won't} take you deep into the internal anatomy
of \TeX{}, nor will it teach you highly efficient but rather obscure
tricks for working with \TeX{}. The objective of this write-up is
to teach you the ``nuts and bolts'' of \TeX\ as quickly as possible so
you can produce the actual documents {\twlit you} need to make with
minimal delay.
\bigskip\par\noindent
As a result, in some cases you'll be shown a way to do something without a
whole lot of explanation or discussion; that format of presentation 
has been adopted to avoid obscuring essential concepts with extraneous
detail. There may be times when I've gone too far toward the minimalist
position, and when that's the case, I urge you to let 
me know: suggestions for the improvement of future versions 
of this write-up are always appreciated.
\vfill\eject
%%%%%%%%%%%%%%%%%%%%%%%%%%%%%%%%%%%%%%%%%%%%%%%%%%%%%%%%%%%%%%%%%%%%%%%%%%%%%%%
\centerline{\twlbf II. ENTERING TEXT
(OTHER THAN TABLES AND EQUATIONS) IN \TeX}
^^{Entering Text}
\bigskip\par\noindent
Every \TeX\ document consists of two parts: actual text, and 
\TeX\ formatting commands.
\bigskip\par\noindent
Most actual text can be entered ``as-is,'' although there are 
some characters which need to be expressed 
specially (either because a particular character isn't
available on your keyboard, or because \TeX\ has designated those
characters for special purposes).
\bigskip\par\noindent
The following section will explain the conventions you need to 
understand in order to be able to enter your document's text in 
\TeX{}'s most popular font, ^{Computer Modern Roman}. Most of the
other fonts you can use with \TeX\ should work in about the same
way, but you may notice some minor differences. 
Thus, until you become more familiar with \TeX{}, you should
probably stick to using the Computer Modern Roman font described
herein.
\bigskip\par\noindent
\centerline{\twlbf Text Which Is Entered Normally in \TeX\ }
\bigskip\par\noindent
Normal text consists of regular letters, numbers, punctuation,
and whitespace,  all of which are entered ``as-is:''
\bigskip\par\noindent
$\bullet$ Lowercase alphabetic letters: 
\bigskip\par\noindent\hglue 0.5truein
{\twltt a b c d e f g h i j k l m n o p q r s t u v w x y z}
\bigskip\par\noindent
$\bullet$ Uppercase alphabetic letters: 
\par\noindent\bigskip\hglue 0.5truein
{\twltt A B C D E F G H I J K L M N O P Q R S T U V W X Y Z}
\bigskip\par\noindent
$\bullet$ The arabic numerals: 
\par\noindent\bigskip\hglue 0.5truein
{\twltt 0 1 2 3 4 5 6 7 8 9}
\bigskip\par\noindent
$\bullet$ Most ^{punctuation}: 
\par\noindent\bigskip\hglue 0.5truein
{\twltt . , ? ! : ; ( ) [ ] ` ' - * / + = @}
\bigskip\par\noindent
$\bullet$ ``^{White-space}'' (^{spaces}, ^{tabs}, and 
^{carriage-returns}). 
\bigskip\par\noindent
\TeX\ will freely condense
or expand white space to suit its typographical preferences,
although there are exceptions to this rule. For instance, 
you can enter a ``^{control space}''
(by saying $\backslash${\twltt \char `\ }), which \TeX\ {\twlit 
will} honor unconditionally as being a ``real'' space which is
not to be adjusted. Insert a tiny space by saying 
{\twltt ^{$\backslash$thinspace}}.
Larger spaces can be forced into text with
{\twltt ^{$\backslash$enspace}} or {\twltt ^{$\backslash$quad}}
or {\twltt ^{$\backslash$qquad}}.
Generally, however, simply hit the space bar to put space between
two words. 
\bigskip\par\noindent
One other note: if you're entering an exceptionally long line
and want to break that text into two separate lines but {\twlbf don't}
want \TeX\ to treat the carriage return at the end of the first
line as being equivalent to a space, put a
{\twltt \%} sign at the end of the first line. \TeX\ will then
concatenate (or ``abut'') the two lines without any visible
break.
^^{concatenating lines}
^^{abutting lines}
\bigskip\bigskip\par\noindent
\centerline{\twlbf ^{Special Characters} in \TeX\ }
\bigskip\par\noindent
Not all characters can be so easily entered in \TeX{}. Some text
must receive special treatment if you want it to come out 
looking right when typeset in \TeX{}. For example:
\bigskip\bigskip\par\noindent
$\bullet$  Most typists are accustomed to using a 
conventional ``androgynous'' ^{double quote mark} ({\twltt "}) at 
both the start and end of a quotation. \TeX{} is a different beast.
Like most typesetting programs, \TeX\ uses different
marks to open and close quotations: on the left side, use  
two successive backslant accent marks ({\twltt `{}`}), 
thereby producing the correct
opening quote mark (``); on the right side of the quotation, use 
two successive single quote marks ({\twltt '{}'}), 
resulting in a proper closing ^{quote mark}
('').\footnote{$^{4}$}{If you absolutely must have
an ^{androgynous double quote mark}, enter
{\twltt $\{\backslash$twltt\ {\twltt "}$\}$}\par}
\bigskip\bigskip\par\noindent
$\bullet$ You should also know how to enter the three different types of
^{dashes} available in TeX: 
the ^{hyphen} (used to form compound words), the
^{en-dash} (used to indicate ranges of items), 
and the ^{em-dash} (used to
signal pauses in discourse)
\bigskip\par\noindent\hglue 0.5truein
-\qquad\qquad\enspace\enspace -- \qquad\qquad ---
\bigskip\par\noindent
which are obtained by entering one, two or three dashes in succession:
\bigskip\par\noindent\hglue 0.5truein
{\twltt
$-$
\qquad\quad\
$-${}$-$
\qquad\quad
$-${}$-${}$-${}
}
\bigskip\bigskip\par\noindent
$\bullet$ You should also know how to make an ^{ellipsis}
properly in \TeX{}.
A properly created \TeX\ ellipsis should look like:
\bigskip\par\noindent\hglue 0.5truein
The tension was $\ldots$ palpable.
\bigskip\par\noindent
Obtain such an ellipsis by saying:
\bigskip\par\noindent\hglue 0.5truein
{\twltt The tension was \${}$\backslash${}ldots\${} palpable.}
\bigskip\bigskip\par\noindent
$\bullet$ Other special characters are the
^{dollar sign}, ^{pound sign}, ^{percent sign}, ^{ampersand sign},
^{underscore}, ^{vertical bar}, ^{less than sign}, ^{greater than sign}, 
and pound sterling sign:
\par\noindent\bigskip\hglue 0.5truein
{\twltt \$ \qquad \# \qquad \% \qquad \&
\qquad \_ \qquad\quad $\mid$  
\quad \qquad\enspace $<$ \qquad $>$ \quad \qquad {\it \$}}
\bigskip\par\noindent
which you can create by saying: 
\par\noindent\bigskip\hglue 0.5truein
{\twltt 
$\backslash$\$\qquad
$\backslash$\#\qquad
$\backslash$\%\qquad
$\backslash$\&\qquad
$\backslash$\_\qquad
\${}$\backslash$mid\$\qquad
\${}$<${}\${}\qquad
\${}$>${}\${}\qquad
$\{$$\backslash${\twltt it\ }$\backslash${}\${}$\}$}
\bigskip\bigskip\par\noindent
$\bullet$ Additional marks which need to be specially constructed 
include the ^{caret} (or ``^{hat}''), ^{tilde}
(or ``^{wave}''), ^{left curly brace}, ^{right curly brace}, 
and ^{backslash}, all of which have special uses in \TeX{}:
\par\noindent\bigskip\hglue 0.5truein
\enspace {\^{ } \qquad\qquad \~{ } \qquad\qquad $\{$
\quad\qquad $\}$ \quad\qquad\qquad $\backslash$}
\bigskip\par\noindent
Get those characters by saying:
\bigskip\par\noindent\hglue 0.5truein
{\twltt
$\backslash{}$% =\              ---- (caret)
\^{ }% =^
$\{$% ={
\ % =space
$\}$% =}
\qquad
%
$\backslash{}$% =\              ---- (tilde)
\~{ }% =~
$\{$% ={
\ % =space
$\}$% =}
\qquad
%
$\$${}% =$            ---- (left curly brace)
$\backslash{}$% =\
$\{$% ={
$\$${}% =$
\qquad 
%
$\$${}% =$            ---- (right curley brace)
$\backslash{}$% =\
$\}$% =}
$\$${}% =$
\qquad
%
$\$${}% =$              ---- (backslash)
$\backslash{}$% =\
backslash% = backslash (text)
$\$${}% =$
\qquad}
%
\bigskip\bigskip\par\noindent
$\bullet$ You can also get many other special symbols, including
a ^{dagger}, ^{double dagger}, ^{copyright symbol}, 
^{paragraph symbol}, 
^{section symbol}, and 
^^{spanish punctuation}
``spanish'' punctuation symbols:
\bigskip\par\noindent\hglue 0.5truein
{\twltt \enspace \dag \qquad\qquad \ddag \qquad\qquad\quad \copyright 
\qquad\qquad\qquad \P \quad\qquad \S \qquad\enspace !` \quad\quad ?`}
\bigskip\par\noindent
Make them in your document by entering:
\bigskip\par\noindent\hglue 0.5truein
{\twltt
$\backslash${}dag
\qquad
$\backslash${}ddag
\qquad
$\backslash${}copyright
\qquad
$\backslash${}P
\qquad
$\backslash${}S
\qquad
!{}`
\quad\enspace
!{}?
}
\bigskip\bigskip\par\noindent
$\bullet$ \TeX{}'s support for foreign languages goes far beyond its 
ability to make upside-down exclamation points and question marks. Unlike
many American document processing systems which do a
pathetic job of handling the special characters common in foreign
languages, \TeX\ handles special ^{foreign characters} easily:
\bigskip\par\noindent\hglue 0.5truein
{\enspace 
\aa \qquad\enspace \AA \qquad\enspace
\ae \qquad\enspace \AE \qquad \l \qquad \L \qquad\enspace
\o \qquad \O \qquad \oe \qquad \OE \qquad \ss
}
\bigskip\par\noindent
Obtain these ^{special foreign characters} (in Computer Modern Roman 
fonts) by saying:
\bigskip\par\noindent\hglue 0.5truein
{\twltt
$\backslash${}aa \quad 
$\backslash${}AA \quad
$\backslash${}ae \quad
$\backslash${}AE \quad
$\backslash${}l  \quad
$\backslash${}L  \quad
$\backslash${}o  \quad
$\backslash${}O  \quad
$\backslash${}oe \quad
$\backslash${}OE \quad
$\backslash${}ss
}
\bigskip\bigskip\par\noindent
$\bullet$ \TeX\ can handle ^{special foreign accents}, too. 
^{Superior accents} include:
\bigskip\par\noindent\hglue 0.5truein
{\quad \`{o} \qquad\enspace\enspace \'{o} \qquad\enspace\enspace
\^{o} \qquad\enspace \"{o} \qquad\enspace\enspace \~{o} 
\qquad\enspace \={o} \qquad\enspace\enspace
\.{o} \qquad\enspace\enspace \u{o} \qquad\enspace
\v{o} \qquad\enspace\enspace \H{o}}
\bigskip\par\noindent
Obtain them by writing:
\bigskip\par\noindent\hglue 0.5truein
{\twltt 
$\backslash$`$\{$o$\}$\quad
$\backslash$'$\{$o$\}$\quad
$\backslash$\^{ }$\{$o$\}$\quad
$\backslash$"$\{$o$\}$\quad
$\backslash$\~{ }$\{$o$\}$\quad
$\backslash$=$\{$o$\}$\quad
$\backslash${.}$\{$o$\}$\quad
$\backslash$u$\{$o$\}$\quad
$\backslash$v$\{$o$\}$\quad
$\backslash$H$\{$o$\}$\quad
}
\bigskip\bigskip\par\noindent
^^{Inferior accents}
$\bullet$ Inferior and special ^{multi-letter tied accents} include:
\bigskip\par\noindent\hglue 0.5truein
{\enspace 
\c{o} \qquad\enspace\enspace
\d{o} \qquad\enspace\enspace  
\b{o} \qquad\enspace\enspace \t{oo}} 
\bigskip\par\noindent
obtained by saying:
\bigskip\par\noindent\hglue 0.5truein
{$\backslash$c$\{$o$\}$\quad 
$\backslash$d$\{$o$\}$\quad 
$\backslash$b$\{$o$\}$\quad
$\backslash$t$\{$oo$\}$}
\bigskip\bigskip\par\noindent
You now know how to enter all the symbols which appear on a standard
computer terminal's keyboard, as well as selected other text symbols.
There are many other symbols available in \TeX\ which you'll normally
need only when setting mathematical equations. We'll talk about them 
later in this write-up. 
\vfill\eject
%%%%%%%%%%%%%%%%%%%%%%%%%%%%%%%%%%%%%%%%%%%%%%%%%%%%%%%%%%%%%%%%%%%%%%%%%%%%%%%
\centerline{\twlbf Structuring The Text You Enter; ^{Making Paragraphs}}
\bigskip\par\noindent
Let's now talk about how you should structure the text you're entering.
Generally speaking, you have a lot of leeway in this area.
\bigskip\par\noindent
Most often, people will elect to enter their text ``normally'' (in
paragraph-shaped pieces), with a blank line between ^{paragraphs}.
Some people, however, prefer to use a ``sentence-by-sentence'' 
format, starting each new sentence on a new line (again leaving 
a blank line between paragraphs). 
\bigskip\par\noindent
If you prefer, you can 
begin paragraphs with the command {\twltt $\backslash$par} 
instead of signalling the beginning of each new paragraph with a blank line.
\bigskip\par\noindent
^^{indentation}
You don't need to manually tab or space to indent the first line of
each paragraph since \TeX\ will automatically indent the beginning of each
paragraph for you.\footnote{$^{5}$}{If you prefer a 
^{block-style format} with un-indented paragraphs and an extra space between 
paragraphs, (i.e., like the style I elected to use in this 
document), you can replace the blank line between paragraphs with 
{\twltt $\backslash$bigskip$\backslash$par$\backslash$noindent} instead.
\bigskip\par\noindent 
You could also set the 
{\twltt $\backslash$parindent} paragraph indentation parameter
to zero, but that can have additional unexpected side effects since
the {\twltt $\backslash$parindent} dimension is used in a number
of different places for diverse purposes. 
Therefore, we recommend you manually suppress 
indentation on a paragraph-by-paragraph basis until you become 
somewhat proficient in the use of \TeX{}.\par}
\bigskip\bigskip\par\noindent
\centerline{\twlbf ^{Comments}}
\bigskip\par\noindent
One habit you may want to acquire is insertion of comments throughout
your document to set off various sections of text or to remind yourself
what a particularly obscure \TeX\ command does. To insert a comment in
a \TeX\ document, simply type a ^{percent sign} and then enter your comment.
The percent sign, and anything which comes after it on
the same line, will be completely
ignored by \TeX{} when it processes your document. For example:
\bigskip\par\noindent\hglue 0.5truein
{\twltt \% Remember to add disclaimers here before final printing!}
\bigskip\par\noindent
Comments in \TeX\ can be entered on their own line (thus making it easy for you
to spot those comments when you look through your raw \TeX\ file), or 
you can enter \TeX\ comments at the end of a line containing
other \TeX\ commands, whichever you prefer.
\bigskip\par\noindent
^^{Missing text in a document}
Note: if a piece of your document is mysteriously missing when you
print it out, look for a ``real'' percent sign in your text which may
have gotten interpreted (incorrectly) as a comment.
\bigskip\bigskip\par\noindent
\centerline{\twlbf ^{Font Size}}
\bigskip\par\noindent
\TeX\ gives you the ability to work with a variety of sizes and styles of 
type within a single document. To automatically define a full set of
Computer Modern Roman fonts, say:
\bigskip\par\noindent\hglue 0.5truein
{\twltt $\backslash$input ^{fontsize.tex}}
\bigskip\par\noindent
near the top of your document.\footnote{$^{6}$}{If you'd like to see
the contents of that font definition command file, say:
\bigskip\par\noindent\hglue 0.5truein
{\twltt \$ TYPE/PAGE TEX\_INPUTS:FONTSIZE.TEX}
\bigskip\par\noindent
Note to \TeX\ users on other machines: a copy of this font definition file
is available for retrieval via anonymous FTP from DECOY.UOREGON.EDU
(128.223.32.19) in pub/tex/samples
\bigskip\par\noindent}
\bigskip\par\noindent
You can then declare a ^{default font size family} for your document by
saying:
\bigskip\par\noindent\hglue 0.5truein
{\twlbf Font size declaration} \qquad {\twlbf Sample}
\medskip\par\noindent\hglue 0.5truein
{\twltt $\backslash$eightpoint\ \ \ \ \ } \qquad\qquad 
{\eightpoint This is eight point type.}
\par\noindent\hglue 0.5truein
{\twltt $\backslash$ninepoint\ \ \ \ \ \ } \qquad\qquad
{\ninepoint This is nine point type.}
\par\noindent\hglue 0.5truein
{\twltt $\backslash$tenpoint\ \ \ \ \ \ \ } \qquad\qquad
{\tenpoint This is ten point type.}
\par\noindent\hglue 0.5truein
{\twltt $\backslash$elevenpoint\ \ \ \ } \qquad\qquad
{\elevenpoint This is eleven point type.}
\par\noindent\hglue 0.5truein
{\twltt $\backslash$twelvepoint\ \ \ \ } \qquad\qquad 
{\twelvepoint This is twelve point type.}
\par\noindent\hglue 0.5truein
{\twltt $\backslash$fourteenpoint\ \ } \qquad\qquad
{\fourteenpoint This is fourteen point type.}
\par\noindent\hglue 0.5truein
{\twltt $\backslash$seventeenpoint\ } \qquad\qquad
{\seventeenpoint This is seventeen point type.}
\bigskip\par\noindent
as appropriate. Usually, you'll want to use 
{\twltt $\backslash$twelvepoint} for most of the copy you'll prepare.
\bigskip\par\noindent
If you're not a typesetter or journalism major, you may not be familiar with 
using ``^{points}'' as a unit of measure. There are 72.27 points to an inch, 
but you are probably better off conceptualizing twelve points as being the 
``bigness'' of the text you're reading right now. 
\bigskip\par\noindent
Ten and eleven point fonts are 
also commonly used, although they may look ``small'' to a reader who's 
accustomed to pica or elite typewritten text.
\bigskip\par\noindent
\vbox{
The ^{Computer Modern (``CM'') font family} consists of:
\bigskip\par\noindent
\settabs 7\columns
\+{\bf Generally}&{\bf 12pt}&{\bf 11pt}&{\bf 10pt}&{\bf Name}&&{\bf  Sample}\cr
\medskip\par\noindent
\+{\tt $\backslash$rm}&{\tt $\backslash$twlrm}&{\tt $\backslash$elvrm}&
{\tt $\backslash$tenrm}&Roman&&{\twlrm AaBbCc}\cr
\+{\tt $\backslash$bf}&{\tt $\backslash$twlbf}&{\tt $\backslash$elvbf}&
{\tt $\backslash$tenbf}&Roman Bold Extended&&{\twlbf AaBbCc}\cr
\+{\tt $\backslash$it}&{\tt $\backslash$twlit}&{\tt $\backslash$elvit}&
{\tt $\backslash$tenit}&Roman Text Italic&&{\twlit AaBbCc}\cr
\+{\tt $\backslash$tt}&{\tt $\backslash$twltt}&{\tt $\backslash$elvtt}&
{\tt $\backslash$tentt}&Roman Typewriter Text&&{\twltt AaBbCc}\cr
\+{\tt $\backslash$sl}&{\tt $\backslash$twlsl}&{\tt $\backslash$elvsl}&
{\tt $\backslash$tensl}&Roman Slanted Roman&&{\twlsl AaBbCc}\cr
\+{\tt $\backslash$mit}&{\tt $\backslash$twli}&{\tt $\backslash$elvi}&
{\tt $\backslash$teni}&Roman Math Italic&&{\twli AaBbCc}\cr
\+&{\tt $\backslash$twlex}&{\tt $\backslash$elvex}&
{\tt $\backslash$tenex}&Roman Math Extension&&{\twltt n/a}\cr
\+{\tt $\backslash$cal}&{\tt $\backslash$twlsy}&{\tt $\backslash$elvsy}&
{\tt $\backslash$tensy}&Roman Math Symbols&&{\twlsy A\ B\ C}\cr}
\bigskip\par\noindent
The majority of this document has been set in Computer Modern Roman 12 
point, with the examples set in CMR Typewriter Text 12 point and 
emphasized text set in CMR Bold Extended 12 point and
CMR Text Italic 12 point. Finally, the titles on the cover page have 
been set in CMR Bold Extended 17 point ({\twltt $\backslash$stnbf}) .
\bigskip\par\noindent
While the fonts we've just described
are probably the only fonts you'll ever need for most documents, 
there are many other fonts which are also available for your use on the VAX.
If you'd like to see all the fonts currently installed on the VAX, 
use the commands:
\par\noindent\bigskip\hglue 0.5truein
{\twltt \$\ DIR TEX\_FONTS:*.*}
\par\noindent\bigskip
For more information about these fonts, see Appendix F of 
{\twlit The \TeX{}book}, or see Volume 3 of Knuth's {\twlit 
Computers and Typesetting.}
\bigskip\par\noindent
One more important point about declaring a default font size: when you 
say {\twltt $\backslash$twelvepoint} (or {\twltt $\backslash$tenpoint} or
whatever), by doing so you are choosing a default font size,
but you are {\twlbf also} setting {\twlit the default spacing between lines!}
^^{line spacing}
If your document's line spacing looks too ``loose'' or too ``tight'', 
check to make sure that the default font size you specified corresponds
to the font size of the majority of the copy you're setting!
\bigskip\bigskip\par\noindent
\centerline{\twlbf ^{Font Style}}
\bigskip\par\noindent
Now that you've got your default font size defined, 
you're ready to tell \TeX\ what
parts of your document should be set in each font. 
To declare a ``default'' font style to be used until you specify 
otherwise, simply enter that
font's name near the top of your document, before you begin entering
document text.  Thus, to use Computer Modern 
Roman fonts (at the current default size) as your default font, you'd enter:
\bigskip\par\noindent\hglue 0.5truein
{\twltt $\backslash$rm}
\bigskip\par\noindent
All the text set from that point forward will be typeset in Computer Modern
Roman at the default font size you've declared.
If you want to set just a few words of text in another font style (such as an
^{italic} or ^{bold}
face font), use curly braces to create a ``^{group}'' limiting
the effect of the font change to the desired section of text:
\bigskip\par\noindent
{\parindent=0.5truein\narrower{\twltt
When one is writing, there are times when one needs
\par\noindent
$\{${}$\backslash$bf interesting$\}$ 
examples, or at least some which
\par\noindent
are $\{${}$\backslash$it moderately relevant$\}$.}\smallskip\par}
\bigskip\par\noindent
This should yield output which looks like:
\bigskip\par\noindent
{\parindent=0.5truein\narrower
When one is writing, there are times when one needs 
{\twlbf interesting} examples, or at least some which are 
{\twlit moderately relevant.}\bigskip\par\noindent}
If you accidentally omit the closing curly brace for a group,
you'll probably end up with pages and pages of text set in the
wrong font, together with an error message from \TeX\ 
stating that you have a ``{\twltt{}$\backslash$^{end at level one}}'', or 
something equally non-explanatory. That's your cue to go back and insert the
closing curly brace you missed.
\vfill\eject
%%%%%%%%%%%%%%%%%%%%%%%%%%%%%%%%%%%%%%%%%%%%%%%%%%%%%%%%%%%%%%%%%%%%%%%%%%%%%%%
\centerline{\twlbf ^{Underlining}}
\bigskip\par\noindent
Underlining is seldom seen in typeset material, because it is a pain
to typeset (using mechanical type), because underlining is typically
replaced with italics in typeset matter as a matter of convention, and
because underlining can interfere with descenders on some letters (see,
for example, the letter ``g'' in the example below). 
\bigskip\par\noindent 
Nonetheless, we realize that there are some circumstances 
(such as during the preparation of a dissertation)
when you have no choice: when you're writing a dissertation you simply 
have to underline some things. \TeX\ can handle that requirement:
\bigskip\par\noindent\hglue 0.5truein
{\twltt I don't need no stinking 
$\backslash$^{underbar}$\{$badge$\}$, Mee--ster!}
\bigskip\par\noindent
This results in output which looks like:
\bigskip\par\noindent\hglue 0.5truein
I don't need no stinking \underbar{badge}, Mee-ster!
\bigskip\bigskip\bigskip\par\noindent
\centerline{\twlbf ^{Line Spacing} 
(^{Double-Spacing}, ^{Skipping A Single 
Line}, etc.)}
\bigskip\par\noindent
To double space text in \TeX{}, issue the command:
\bigskip\par\noindent\hglue 0.5truein
{\twltt $\backslash$baselineskip$=$2$\backslash$normalbaselineskip}
\bigskip\par\noindent
Text set by \TeX\ in double-spaced form looks like:
\medskip\par\noindent
{\parindent=0.5truein\narrower{
\baselineskip=2\baselineskip
The use of a double spaced format can do much to make the 
inconsequential appear weighty, and the weighty, ponderous. 
It also is convenient when you need to correct those
errors introduced by your colleagues.
\bigskip\bigskip\par\noindent}}
\baselineskip=\normalbaselineskip
Return to normal spacing by saying:
\bigskip\par\noindent\hglue 0.5truein
{\twltt $\backslash$^{baselineskip}=$\backslash$^{normalbaselineskip}}
\bigskip\bigskip\bigskip\par\noindent
To insert just one ``extra'' blank line at a particular
spot in your document, enter: 
\bigskip\par\noindent\hglue 0.5truein
{\twltt $\backslash$^{bigskip}}
\bigskip\par\noindent
To get just an extra {\twlit half}-line of space say:
{\twltt$\backslash$^{medskip}}
\bigskip\par\noindent
To get just an extra {\twlit quarter}-line of extra space say:
{\twltt $\backslash$^{smallskip}}
\vfill\eject
%%%%%%%%%%%%%%%%%%%%%%%%%%%%%%%%%%%%%%%%%%%%%%%%%%%%%%%%%%%%%%%%%%%%%%%%%%%%%%%
\centerline{\twlbf ^{Block Quotations}}
\bigskip\par\noindent
If you are quoting lengthy passages from a work, you'll normally want
to indent left and right 
and single space the quotation in block format. To do that 
in \TeX\ you'd say:
\bigskip\par\noindent\hglue 0.5truein
{\twltt $\{${}$\backslash$narrower\ $\{${\twlit{}quotation text goes here}$\}$%
\ $\backslash$bigskip$\backslash$par$\}$}
\bigskip\par\noindent
Some notes on this process of handling block quotations: 
\bigskip\par\noindent
(1) You {\twlbf must}
end the paragraph before you exit the group containing the 
{\twltt $\backslash$^{narrower}} command; if you fail to 
do so, the paragraph will revert to its normal width as
if you hadn't entered a {\twltt $\backslash$narrower} command
at all. (That is why the 
{\twltt $\backslash$bigskip$\backslash$par} command is
contained within the curly braces.)
\bigskip\par\noindent 
(2) The block quotation will be indented by the current value of the
{\twltt $\backslash$parindent} parameter. If your default 
{\twltt $\backslash$parindent} is too large or too small for your
requirements, you may need to manually reset it as part of the 
{\twltt $\backslash$narrower} command. For example, to indent 1" try:
\bigskip\par\noindent\hglue 0.5truein
{\twltt $\{${}$\backslash$parindent=1.0in
$\backslash$narrower\ $\{${\twlit{}quotation text goes here}$\}$\  
$\backslash$bigskip$\backslash$par{}$\}$}
\bigskip\par\noindent
(3) Finally, if you are double spacing the rest of your document, you'll
need to explicitly force the block quotation to be single spaced.
Try something like:
\bigskip\par\noindent\hglue 0.5truein
{\twltt $\{${}$\backslash$parindent=1.0in
$\backslash$narrower\ $\backslash$baselineskip=$\backslash$normalbaselineskip
\par\noindent\hglue 0.5truein
$\{${\twlit{}quotation text goes here}$\}$
$\backslash$bigskip$\backslash$par{}$\}$}
\bigskip\par\noindent
For example, here's a small sample block quotation:
\bigskip\par\noindent
{\parindent=0.5truein\narrower
{\twltt$\{${}$\backslash$parindent=0.5in
$\backslash$narrower
$\backslash$baselineskip=$\backslash$normalbaselineskip
\par\noindent
{$\{$The terrible thing about our time is precisely the ease with
\par\noindent
which theories can be put into practice. The more perfect, the
\par\noindent
more idealistic the theories, the more dreadful is their
\par\noindent
realization. We are at last beginning to rediscover what 
\par\noindent
perhaps men knew better in very ancient times, in primitive
\par\noindent
times before utopias were thought of: that liberty is bound up
\par\noindent
with imperfection, and that limitations, imperfections, errors
\par\noindent
are not only unavoidable but salutary. Merton$\}$%
$\backslash$bigskip$\backslash$par{}$\}$}
\bigskip\par\noindent}}
When processed by \TeX{}, that block quotation would look like:
\bigskip\par\noindent
{\parindent=0.5in
\baselineskip=\normalbaselineskip
\narrower
{The terrible thing about our time is precisely 
the ease with which theories can be put into 
practice. The more perfect, the more idealistic the 
theories, the more dreadful is their realization.  
We are at last beginning to rediscover what perhaps 
men knew better in very ancient times, in primitive  
times before utopias were thought of: that liberty 
is bound up with imperfection, and that limitations, 
imperfections, errors are not only unavoidable
but salutary. Merton}
\bigskip\par}
Under some circumstances, you may only want to indent one side of the
block or the other, but not both. When that's the case, use
\bigskip\par\noindent\hglue 0.5truein
{\twltt $\{${}$\backslash$leftskip=1.0in 
\par\noindent\hglue 0.5truein
$\{$Here is some text to be indented from the left margin.$\}$
\par\noindent\hglue 0.5truein
$\backslash$bigskip$\backslash$par$\}$} 
\bigskip\par\noindent
or 
\bigskip\par\noindent\hglue 0.5truein
{\twltt $\{${}$\backslash$rightskip=1.0in 
\par\noindent\hglue 0.5truein
$\{$Here is some text to be indented from the right margin.$\}$
\par\noindent\hglue 0.5truein
$\backslash$bigskip$\backslash$par$\}$} 
\bigskip\par\noindent
You can also indent a single line simply by inserting some horizontal glue:
\bigskip\par\noindent\hglue 0.5truein
{\twltt $\backslash$par$\backslash$noindent$\backslash$hglue 1.0in
\par\noindent\hglue 0.5truein
Here's some text to be indented an inch.
\par\noindent\hglue 0.5truein
$\backslash$par$\backslash$noindent}
\bigskip\bigskip\bigskip\par\noindent
\centerline{\twlbf ^{Centering Text} (For Headings, etc.)}
\bigskip\par\noindent
There are times when you'd like to center some text on the page
for headings. To do this in \TeX{}, use the command:
\bigskip\par\noindent\hglue 0.5truein
{\twltt $\backslash$centerline$\{$text to be centered goes here$\}$}
\bigskip\par\noindent
To set a single line flush left, or a single line flush
right, use the commands:
\bigskip\par\noindent\hglue 0.5truein
{\twltt $\backslash$leftline$\{$text to be set ^{flush left} goes here$\}$}
\bigskip\par\noindent\hglue 0.5truein
{\twltt $\backslash$rightline$\{$text to be set ^{flush right} goes here$\}$}
\bigskip\par\noindent
Note that you can achieve the same effects using more primitive
commands by saying:
\bigskip\par\noindent\hglue 0.5truein
{\twltt $\backslash$par $\backslash$^{hfill} $\{$text to be centered goes 
here$\}$ $\backslash$hfill $\backslash$par}
\bigskip\par\noindent\hglue 0.5truein
{\twltt $\backslash$par $\{$text to be set flush left goes here$\}$ $\backslash$hfill $\backslash$par}
\bigskip\par\noindent\hglue 0.5truein
{\twltt $\backslash$par
$\backslash$hfill $\{$text to be set flush right goes here$\}$ $\backslash$par}
\bigskip\par\noindent
This more primitive approach, using {\twltt $\backslash$hfill} commands, will
particularly come in handy later when you begin to work with headers and
footers, and under other circumstances when the more elegant {\twltt
$\backslash$centerline} or {\twltt $\backslash$leftline} or 
{\twltt $\backslash$rightline} approaches are infeasible.
\vfill\eject
%%%%%%%%%%%%%%%%%%%%%%%%%%%%%%%%%%%%%%%%%%%%%%%%%%%%%%%%%%%%%%%%%%%%%%%%%%%%%%%
\centerline{\twlbf ^{Footnotes}}
\bigskip\par\noindent
To create a footnote in \TeX{}, use a command such as the following:
\bigskip\par\noindent\hglue 0.5truein
{\twltt Another fascinating condiment is the pink
\par\noindent\hglue 0.5truein
peppercorn.$\backslash$footnote$\{${}\${}\^{}$\{$13$\}${}\${}$\}${}$\{$See, % 
for example, William Poundstone, 
\par\noindent\hglue 0.5truein
$\backslash$it$\{$Big Secrets,$\}$ 
Quill, New York, 1983, pages 26-27.$\backslash$par$\}$}
\par\noindent\bigskip
Some notes about footnotes:
\bigskip\par\noindent
(1) Footnotes are not automatically numbered as they're created. 
Rather, you must manually assign a number (or symbol) to each 
footnote you make. If you rearrange your footnotes, or insert
new footnotes, you'll also need to manually adjust the numbers of existing
footnotes.\footnote{$^{7}$}{See the solution to problem 15.12 in 
Appendix A of Knuth's {\twlit The \TeX{}book} for one way of 
overcoming the limitations associated with manually numbered footnotes.}
\bigskip\par\noindent
(2) By default, \TeX\ builds its footnotes at the bottom of the page
in a rather ugly format. 
If you'd like to fix the footnote production mechanism so your footnotes 
come out looking like the ones used in this write-up, instead, 
put the following commands at the top of your document:
{\twltt 
\bigskip\par\noindent\hglue 0.5truein
$\backslash$parindent=0pt
$\backslash$global$\backslash$skip$\backslash$footins=24pt plus 24pt minus 24pt
\par\noindent\hglue 0.5truein
$\backslash$def$\backslash$footnoterule$\{${}$\backslash$vfil 
$\backslash$vglue 12pt $\backslash$hrule width 2in 
$\backslash$vglue 12pt$\}$}
\bigskip\par\noindent
Note that this footnote fix {\twlbf does} tinker with the {\twltt 
$\backslash$parindent} parameter which is used for a whole host of 
other purposes! Therefore, determine that you truly can't abide the
default footnote format before you decide to employ the fix shown above.
\bigskip\par\noindent
(3) If you must have ^{endnotes} (rather than footnotes), simply insert 
your footnote reference number in the right place, but don't actually use the 
{\twltt $\backslash$footnote} command. Enter the body of your endnote 
at the {\twlbf end} of your document, including the corresponding reference number.
For example:
\bigskip\par\noindent\hglue 0.5truein
{\twltt Another fascinating condiment is the pink
%\par\noindent\hglue 0.5truein
peppercorn.\${}\^{}$\{$13$\}${}\$
\bigskip\par\noindent\hglue 0.5truein
* \quad * \quad *
\bigskip\par\noindent\hglue 0.5truein
$\backslash$vfill$\backslash$eject
\par\noindent\hglue 0.5truein
$\backslash$centerline$\{${}$\backslash$bf ENDNOTES$\}$
\bigskip\par\noindent\hglue 0.5truein
* \quad * \quad *
\bigskip\par\noindent\hglue 0.5truein
$\backslash$bigskip$\backslash$par$\backslash$noindent
\par\noindent\hglue 0.5truein
13. See, for example, William Poundstone, 
$\backslash$it$\{$Big Secrets,$\}$
\par\noindent\hglue 0.5truein
Quill, New York, 1983, pages 26-27.
\par\noindent\hglue 0.5truein
$\backslash$bigskip$\backslash$par$\backslash$noindent}
\bigskip\bigskip\par\noindent
\centerline{\twlbf ^{Headers} and ^{Page Numbers}}
\bigskip\bigskip\par\noindent
\TeX\ will automatically number each page (including the first page)
at the bottom center of the page. However, you can suppress all page
numbers by including the following near the top of your document:
\bigskip\par\noindent\hglue 0.5truein
{\twltt $\backslash$^{nopagenumbers}}
\bigskip\par\noindent
Many people, however, {\twlbf do} like to have the pages of their document numbered.
Typically, the preferred location for those page numbers is the upper right
hand corner of the page.
\bigskip\par\noindent
One way to get page numbers 
in the upper right hand corner of the page
in \TeX\ is to create an appropriate 
{\twltt $\backslash$headline}. For example, to simply put the
page number in the top right corner of each page, enter (near the top
of your document):
\bigskip\par\noindent\hglue 0.5truein
{\twltt $\backslash$headline=$\{${}$\backslash$rm{}$\backslash$hfill%
$\backslash$folio$\backslash$voffset=2$\backslash$normalbaselineskip$\}$}
\bigskip\par\noindent
In the preceding command, {\twltt $\backslash$rm} insures
that our header will be printed in
a normal Roman font, {\twltt $\backslash$folio} is \TeX{}'s symbolic
name for the current page number, and the {\twltt $\backslash$hfill}
will force the page number all the way to the right. By
setting {\twltt $\backslash$voffset} to 
{\twltt 2$\backslash$normalbaselineskip},
we insure that a blank line is left between the page number and the
page's text.
\bigskip\par\noindent
If you want a running section heading as well as a title, 
enter something like:
\bigskip\par\noindent\hglue 0.5truein
{\twltt 
$\backslash$headline=$\{${}$\backslash$rm Summary $\backslash$hfill%
$\backslash$folio$\backslash$voffset=2$\backslash$normalbaselineskip$\}$}
\bigskip\par\noindent
Note that whatever heading happens to be defined when \TeX\ comes to the
{\twlbf end} of a page is the heading that page will receive. 
The critical
thing is what's defined when the page is {\twlbf finished}, not what's
defined when the page is {\twlbf begun}!
\bigskip\par\noindent
A similar policy holds for page numbering. The value of the page number
counter ({\twltt $\backslash$count0}) at the {\twlbf end} of the page
determines the page number of that page, {\twlbf not} the value the page
number counter may have started at. To change the ^{page counter}, say: 
\bigskip\par\noindent\hglue 0.5truein
{\twltt $\backslash$count0={\twlit{}newpagenumber}} 
\bigskip\par\noindent 
somewhere on the page you want renumbered.
\bigskip\par\noindent
If you need ^{lowercase roman numeral page numbers}
^^{roman numeral page numbering} 
^^{prefatory matter page numbering}
(as you might for
prefatory matter), set {\twltt $\backslash$count0} to a negative
value. One way to negate {\twltt $\backslash$count0} is by using the
command:
\bigskip\par\noindent\hglue 0.5truein
{\twltt
$\backslash$ifnum$\backslash$count0$>$0$\backslash$multiply$\backslash$count0
by -1$\backslash$fi}
\bigskip\par\noindent
Alternatively, assuming you know the exact page number at which you want to begin lowercase roman numeral page numbers, you could simply say:
\bigskip\par\noindent\hglue 0.5truein
{\twltt $\backslash$count0=-{\twlit{}newpagenumber}} 
\bigskip\bigskip\par\noindent
\centerline{\twlbf ^{Page Size}; ^{Margins}}
\bigskip\par\noindent
^^{default page size}
To set the size of your \TeX\ document, use commands like:
\bigskip\par\noindent\hglue 0.5truein
{\twltt $\backslash$^{hsize}=7in}
\par\noindent\hglue 0.5truein
{\twltt $\backslash$^{vsize}=9in}
\bigskip\par\noindent
near the top of your document. That will set the size of the text that \TeX\
creates to be seven inches wide and nine inches high, for example.
\bigskip\par\noindent
By default, \TeX\ starts out indenting 1''
from the left hand side of the page, and 1'' down from the top of the
page. If you wanted to increase the left margin by 0.5'', 
and you wanted to decrease the top margin by 0.25'', 
for example, you'd enter:
^^{default margins}
\bigskip\par\noindent\hglue 0.5truein
{\twltt $\backslash$hoffset=0.5in}
\par\noindent\hglue 0.5truein
{\twltt $\backslash$voffset=-0.25in}
\bigskip\par\noindent
You set the right hand (and bottom) margins implicitly when
you set the size of your \TeX\ document and the width of the 
left and top margins.
\bigskip\bigskip\par\noindent
\centerline{\twlbf ^{Leaving Space for Insertions}; 
^{Forcing Page Breaks}}
^^{Insertions}
^^{Page Breaks}
^^{Ejecting a Page}
^^{Top of Page}
\bigskip\par\noindent
There will be times when you want to take explicit control over the 
way \TeX\ spaces paragraphs on the printed page. That is, you might 
want to pop to the top of a new page before beginning a new section, 
or you might want to leave a couple of inches of blank space so that
you have room to tip in a photograph or other illustration.
\bigskip\par\noindent
To finish the current page and go to top of a new page, enter:
\bigskip\par\noindent\hglue 0.5truein
{\twltt $\backslash$vfill$\backslash$eject}
\bigskip\par\noindent
If all you want to do is reserve a couple of inches of blank space
for an illustration, try:
\bigskip\par\noindent\hglue 0.5truein
{\twltt $\backslash$vglue\ 3.0in$\backslash$par$\backslash$noindent}
\bigskip\par\noindent
Be sensitive to the possibility that the three (or however many inches)
of space you request in this way may be provided to you partially on 
one page, and partially on the following page if there isn't enough
room on the current page. If you need to insure you'll get a single 
undivided block of space of the size required, say:
\bigskip\par\noindent\hglue 0.5truein
{\twltt $\backslash$^{vbox}$\{${}$\backslash$vglue 
3.0in$\backslash$par$\backslash$noindent$\}$}
\bigskip\bigskip\par\noindent
\centerline{\twlbf Ending Your \TeX\ Document}
^^{Ending Your TeX Document}
\bigskip\par\noindent
Signal the end of your \TeX\ document by entering:
\bigskip\par\noindent\hglue 0.5truein
{\twltt $\backslash$vfill$\backslash$eject$\backslash$end}
\bigskip\par\noindent
as the very last line of your \TeX\ command file.
\vfill\eject
%%%%%%%%%%%%%%%%%%%%%%%%%%%%%%%%%%%%%%%%%%%%%%%%%%%%%%%%%%%%%%%%%%%%%%%%%%%%%%%
\centerline{\twlbf III. ^{TYPESETTING TABULAR MATERIAL}}
\bigskip
\centerline{\twlbf ^{Using Tabs}}
\bigskip\par\noindent
There may be times when you want to use ``tabs'' in \TeX{}, just as
you might on a typewriter. The first step in using tabs in \TeX\ 
is to set your ``tabstops.'' While there are many ways to do this,
the two most generally useful approaches to doing this are the
evenly-spaced tab stop setting procedure, and the specific-spacing
tab stop setting procedure.
\bigskip\par\noindent
For example, to create eight ^{evenly spaced tab stops}, you'd enter:
\bigskip\par\noindent\hglue 0.5truein
{\twltt $\backslash$settabs 8$\backslash$columns}
\bigskip\par\noindent
If you need to create ^{unevenly spaced tab stops}, try something like
the following:
\bigskip\par\noindent\hglue 0.5truein
{\twltt $\backslash$settabs$\backslash$+%
$\backslash$hglue 1.75in\&%
$\backslash$hglue 0.4in\&%
$\backslash$hglue 1.4in\&%
$\backslash$hglue 1.0in$\backslash$cr}
\bigskip\par\noindent
There's a third way of setting tab stops which you may like even 
more than either of the above two approachs. Essentially, \TeX\ 
can automatically determine the right width for each column of 
your copy based on the effective width of the widest element
to be put in each column. For example, let's assume you wanted to
set up a little summary table describing some common stocks: 
\bigskip\par\noindent\hglue 0.5truein
{\twltt $\backslash$settabs$\backslash$+%
Ticker Symbol$\backslash$qquad\&%
Turgekowski Steamship Service$\backslash$qquad\&
\par\noindent\hglue 0.5truein
999,999$\backslash$qquad\&%
$\backslash$\$9,999.99$\backslash$qquad\&%
$\backslash$\$9,999.99$\backslash$qquad\&%
$\backslash$\$9,999.99$\backslash$cr}
\bigskip\par\noindent
Note that we've defined the width of the first column by writing its
heading (and a trailing {\twltt $\backslash$qquad} worth of space)
because that is the widest element which will be present in that column.
For the second column, the longest element will be one particularly long
name of an actual firm, together with a little additional space.
The remaining columns, which will be used to report stock trading volume
and high, low, and closing stock prices, were all created using dummy
``worst-case'' numeric values (plus a little extra space).
\bigskip\par\noindent
Regardless of the tab-setting approach you elect to employ, once you've
created your tab stops, you can easily set type at those tab spots by
saying something like:
\bigskip\par\noindent\hglue 0.5truein
{\twltt $\backslash$+Blah\&Foo\&Phft\&\&Thdd$\backslash$cr
\par\noindent\hglue 0.5truein
$\backslash$+ABC\&DEFGH\&IJ\&KLMNOP$\backslash$cr}
\bigskip\par\noindent
I.E., each tabbed line begins with a {\twltt $\backslash$+} and ends with a
{\twltt $\backslash$cr}. Between those delimiters, you indicate 
you want to go to the next tab stop by inserting an ampersand 
({\twltt \&}). Insert two successive ampersands to skip a column. Note that
you do not need to ``use'' all tabstops on each tabbed line before ending
that line --- you can end
a tabbed line with {\twltt $\backslash$cr} after using only one (or even
{\twlbf none}) of the tabstops which you've defined.
\bigskip\par\noindent
If you ever need to un-define your tabstops, enter:
\bigskip\par\noindent\hglue 0.5truein
{\twltt $\backslash$cleartabs}
\bigskip\bigskip\par\noindent
\centerline{\twlbf ^{Typesetting Formal Ruled Tables}}
^^{Formal Ruled Tables}
^^{Ruled Tables}
^^{Tables}
\bigskip\par\noindent
\TeX\ can also set more complicated ``formal'' ruled tables. However,
before you decide to try doing this, a warning: setting tables
in \TeX\ can be rather tricky.
Consider this example:
\bigskip\par\noindent
{\vbox{\offinterlineskip
\halign{
\hglue 0.5in\strut\vrule#&
\twlbf\hfil\quad#\quad\hfil&
\vrule#&
\twlbf\hfil\quad#\quad\hfil&
\vrule#&
\hfil\quad#\quad\hfil&
\vrule#&
\hfil\quad#\quad\hfil&
\vrule#\cr
% template ends; now set the headings
\multispan{9}\hglue 0.5in\vrule\hrulefill\vrule\cr
&&\omit&&\omit&&\omit&&\cr
\multispan{9}\hglue 0.5in\vrule\hfill\twlbf SOME SERVICE PISTOLS\hfill\vrule\cr
&&\omit&&\omit&&\omit&&\cr
\multispan{9}\hglue 0.5in\vrule\hrulefill\vrule\cr
&&&&&&&&\cr
&\twlbf Manufacturer&&\twlbf Model&&\twlbf Caliber&&\twlbf Capacity&\cr
&&&&&&&&\cr
% headings end; now set the body of the table
\multispan{9}\hglue 0.5in\vrule\hrulefill\vrule\cr
&&&&&&&&\cr
&Glock&&17&&9mm Para&&19&\cr
&Sig-Sauer&&P226&&9mm Para&&15&\cr
&Smith\&Wesson&&5926&&9mm Para&&15&\cr
&&&&&&&&\cr
&Glock&&21&&.45 ACP&&13&\cr
&Sig-Sauer&&P220&&.45 ACP&&7&\cr
&Smith\&Wesson&&4506&&.45 ACP&&8&\cr
&&&&&&&&\cr
\multispan{9}\hglue 0.5in\vrule\hrulefill\vrule\cr
}}}
\bigskip\par\noindent
The actual \TeX\ commands used to create that table look like:
%%%%%%%%%%%%%%%%%%%%%%%%%% table text begins here %%%%%%%%%%%%%%%%%%%%%%%%%%%%
\bigskip\par\noindent\hglue 0.5truein
{\twltt 
$\{$$\backslash$vbox$\{${}$\backslash$offinterlineskip$\backslash$halign$\{$
\par\noindent\hglue 0.5truein
$\backslash$hglue 0.5in$\backslash$strut$\backslash$vrule\#\&
\par\noindent\hglue 0.5truein
$\backslash$bf$\backslash$hfil$\backslash$quad\#$\backslash$quad$\backslash$hfil\&%
$\backslash$vrule\#\&
\par\noindent\hglue 0.5truein
$\backslash$bf$\backslash$hfil$\backslash$quad\#$\backslash$quad$\backslash$hfil\&%
$\backslash$vrule\#\&
\par\noindent\hglue 0.5truein
$\backslash$hfil$\backslash$quad\#$\backslash$quad$\backslash$hfil\&%
$\backslash$vrule\#\&
\par\noindent\hglue 0.5truein
$\backslash$hfil$\backslash$quad\#$\backslash$quad$\backslash$hfil\&%
$\backslash$vrule\#$\backslash$cr
\par\noindent\hglue 0.5truein
\% template ends; now set the headings
\par\noindent\hglue 0.5truein
$\backslash$multispan$\{$9$\}${}$\backslash$hglue 0.5in$\backslash$vrule$\backslash$hrulefill$\backslash$vrule$\backslash$cr
\par\noindent\hglue 0.5truein
\&\&$\backslash$omit\&\&$\backslash$omit\&\&$\backslash$omit\&\&$\backslash$cr
\par\noindent\hglue 0.5truein
$\backslash$multispan$\{$9$\}${}$\backslash$hglue 0.5in$\backslash$vrule$\backslash$hfill$\backslash$bf SOME SERVICE PISTOLS
\par\noindent\hglue 0.5truein
$\backslash$hfill$\backslash$vrule$\backslash$cr
\par\noindent\hglue 0.5truein
\&\&$\backslash$omit\&\&$\backslash$omit\&\&$\backslash$omit\&\&$\backslash$cr
\par\noindent\hglue 0.5truein
$\backslash$multispan$\{$9$\}${}$\backslash$hglue 0.5in$\backslash$vrule$\backslash$hrulefill$\backslash$vrule$\backslash$cr
\par\noindent\hglue 0.5truein
\&\&\&\&\&\&\&\&$\backslash$cr
\par\noindent\hglue 0.5truein
\&$\backslash$bf Manufacturer\&\&$\backslash$bf Model\&\&
\par\noindent\hglue 0.5truein
$\backslash$bf Caliber\&\&$\backslash$bf Capacity\&$\backslash$cr
\par\noindent\hglue 0.5truein
\&\&\&\&\&\&\&\&$\backslash$cr
\par\noindent\hglue 0.5truein
\% headings end; now set the body of the table
\par\noindent\hglue 0.5truein
$\backslash$multispan$\{$9$\}${}$\backslash$hglue 0.5in$\backslash$vrule$\backslash$hrulefill$\backslash$vrule$\backslash$cr
\par\noindent\hglue 0.5truein
\&\&\&\&\&\&\&\&$\backslash$cr
\par\noindent\hglue 0.5truein
\&Glock\&\&17\&\&9mm Para\&\&19\&$\backslash$cr
\par\noindent\hglue 0.5truein
\&Sig-Sauer\&\&P226\&\&9mm Para\&\&15\&$\backslash$cr
\par\noindent\hglue 0.5truein
\&Smith$\backslash$\&Wesson\&\&5926\&\&9mm Para\&\&15\&$\backslash$cr
\par\noindent\hglue 0.5truein
\&\&\&\&\&\&\&\&$\backslash$cr
\par\noindent\hglue 0.5truein
\&Glock\&\&21\&\&.45 ACP\&\&13\&$\backslash$cr
\par\noindent\hglue 0.5truein
\&Sig-Sauer\&\&P220\&\&.45 ACP\&\&7\&$\backslash$cr
\par\noindent\hglue 0.5truein
\&Smith$\backslash$\&Wesson\&\&4506\&\&.45 ACP\&\&8\&$\backslash$cr
\par\noindent\hglue 0.5truein
\&\&\&\&\&\&\&\&$\backslash$cr
\par\noindent\hglue 0.5truein
$\backslash$multispan$\{$9$\}${}$\backslash$hglue 0.5in$\backslash$vrule$\backslash$hrulefill$\backslash$vrule$\backslash$cr
\par\noindent\hglue 0.5truein
$\}\}\}$
}
\bigskip\par\noindent
The {\twltt $\backslash$halign} command used to set that table differs
from the tabbed environment described in the previous section 
in three important ways.
^^{Differences Between Tabbing and Formal Tables} 
One difference between the two table construction approaches 
is that \TeX\ itself determines the required width of each 
column in the {\twltt $\backslash$halign} approach without the need for 
you to manually intervene in an effort to guess the widest element present 
in each column. 
A second difference, which happens to be a 
disadvantage of the {\twltt $\backslash$halign} approach,
is that {\twltt $\backslash$halign} doesn't yield constant-width 
columns on a multiple table basis unless you employ some ``tricks''; 
the {\twltt $\backslash$+} tabbing approach is much better 
than {\twltt $\backslash$halign}
when you need to generate a whole series of tables with each table having 
the same width columns. 
The third difference between the two methodologies
is that tabbed tables are processed line-by-line, and thus 
don't make \TeX\ work
very hard; {\twltt $\backslash$halign} tables, on the other hand, have to be
examined in their entirety by \TeX\ 
before it can decide how to process them.
Naturally, this can be quite computationally expensive for large tables!
Experience will gradually help you learn to pick the right table-building
tool.
\bigskip\bigskip\par\noindent
\leftline{\twlbf Beginning to Decode the Table-Building Commands}
\bigskip\par\noindent
^^{Decoding The Table Building Commands}
But what do all the {\twltt $\backslash$halign} table building commands mean? 
\bigskip\par\noindent
Begin by conceptualizing the table as nine
columns. Four of the columns are ``real'' columns containing
manufacturer, model, caliber, and capacity information. The other
five columns are ``columns'' in name only: they are actually used
solely to hold the the vertical rules (lines) separating the real columns
of data.
\bigskip\par\noindent
The first line of \TeX\ commands used to produce our table looks like:
\bigskip\par\noindent\hglue 0.5truein
{\twltt $\{$$\backslash$vbox$\{${}$\backslash$offinterlineskip$\backslash$halign$\{$}
\bigskip\par\noindent
In that line of commands:
\bigskip\par\noindent
{\leftskip=0.5truein{The {\twltt $\backslash$vbox}
makes sure the entire table is treated as an indivisible 
organic entity, thereby insuring that the table isn't accidentally split
across two pages.
\bigskip\par\noindent
The {\twltt $\backslash$offerinterlineskip}
prohibits \TeX\ from automatically vertically skipping {\twltt $\backslash$baselineskip} 
every time it begins a new line. It is important to keep 
\TeX\ from doing this when 
you are setting tables which include ``lines'' consisting solely 
of thin horizontal rules, as our table does.
\bigskip\par\noindent
Finally, the {\twltt $\backslash$halign} actually starts 
our horizontally aligned \TeX\ table.
\par\noindent}}
\vfill\eject
%%%%%%%%%%%%%%%%%%%%%%%%%%%%%%%%%%%%%%%%%%%%%%%%%%%%%%%%%%%%%%%%%%%%%%%%%%%%%%%
\leftline{\twlbf ^{Table Template}}
\bigskip\par\noindent
The next five lines of commands, which look like:
{\twltt
\bigskip\par\noindent\hglue 0.5truein
$\backslash$hglue 0.5in$\backslash$strut$\backslash$vrule\#\&%
\par\noindent\hglue 0.5truein
$\backslash$bf$\backslash$hfil$\backslash$quad\#$\backslash$%
quad$\backslash$hfil\&$\backslash$vrule\#\&%
\par\noindent\hglue 0.5truein
$\backslash$bf$\backslash$hfil$\backslash$quad\#$\backslash$%
quad$\backslash$hfil\&$\backslash$vrule\#\&%
\par\noindent\hglue 0.5truein
$\backslash$hfil$\backslash$quad\#$\backslash$quad$\backslash$%
hfil\&$\backslash$vrule\#\&%
\par\noindent\hglue 0.5truein
$\backslash$hfil$\backslash$quad\#$\backslash$quad$\backslash$%
hfil\&$\backslash$vrule\#$\backslash$cr}
\bigskip\par\noindent
define the {\twlbf default format} for the nine columns
comprising each line of the table. 
\bigskip\par\noindent
The declared default format for the first column extends 
from the start of the commands
up to the 1st ampersand; the format for the second column goes from the 1st
ampersand to the 2nd ampersand; the format for the third column goes
from the 2nd to the 3rd; etc. 
\bigskip\par\noindent
Specifically, in our case, the leftmost
column (and thus the entire table) will be (by default):
\bigskip\par\noindent\hglue 0.5truein
Indented {\twltt 0.5}'' from the left margin ({\twltt $\backslash$hglue 0.5in}), and be of
\bigskip\par\noindent\hglue 0.5truein
``Normal'' height ({\twltt $\backslash$strut}).
\bigskip\par\noindent
The leftmost column is also defined to contain
%\bigskip\par\noindent\hglue 0.5truein
a thin vertical line ({\twltt $\backslash$vrule}). 
\bigskip\par\noindent
The pound sign ({\twltt \#}) in that line 
represents the spot where the ``contents'' 
of a column should be added when we begin processing 
the body of the table.\footnote{$^{8}$}{In the case 
of the columns reserved for vertical rules,
we'll never be adding any additional columnular material, but \TeX\ nonetheless
demands that we include a pound sign for each column as a matter of form.
\bigskip\par\noindent}
\bigskip\par\noindent
The second column of the table 
is our first ``real'' column. The text in that column will 
be set in twelve point bold face type ({\twltt $\backslash$bf}), 
and centered (notice the two {\twltt $\backslash$hfil}s), with a 
minimum of one {\twltt $\backslash$quad} worth of space on each side 
of the text we'll eventually supply.\footnote{$^{9}$}{Omission
of the {\twltt $\backslash$quad}s would result in the lines of the table 
directly abutting the edges of the widest entry in each column --- an 
aesthetically undesirable occurrence.}
\bigskip\par\noindent
The third column of the template consists solely of a 
{\twltt $\backslash$vrule}
and a ``dummy'' pound sign provided exclusively to placate \TeX{}.
\bigskip\par\noindent
The remaining six column definitions are similar to those already discussed.
Note, though, that the last (ninth) column ends with a {\twltt $\backslash$cr},
not another ampersand. 
\bigskip\par\noindent
We're now done defining the template \TeX\ needed for the body of our table.
\bigskip\bigskip\par\noindent
\leftline{\twlbf ^{Table Headings}}
\bigskip\par\noindent
The first non-template/``output-producing'' table line looks like:
\bigskip\par\noindent\hglue 0.5truein
{\twltt $\backslash$multispan$\{$9$\}${}$\backslash$hglue 0.5in$\backslash$vrule$\backslash$hrulefill$\backslash$vrule$\backslash$cr}
\bigskip\par\noindent
It is designed to create the top horizontal line of the table.
\bigskip\par\noindent
The {\twltt $\backslash$multispan$\{$9$\}$} command tells \TeX\ that we
want to enter some stuff which will cross (or span) nine columns of the table.
The rest of that command says ``skip a half inch horizontally, then 
put in a thin vertical rule, then fill the rest of the line with a thin horizontal
rule, putting in another thin vertical rule at the right hand margin. 
End this line.'' 
\bigskip\par\noindent
The {\twltt $\backslash$multispan} command automatically
cancels the commands present in the default table template, including 
the {\twltt $\backslash$strut} (and all the other good stuff) so that 
the first  ``line'' of our table will actually be no taller than
the height of our thin rules. 
\bigskip\par\noindent
The next line of commands:
\bigskip\par\noindent\hglue 0.5truein
{\twltt \&\&$\backslash$omit\&\&$\backslash$omit\&\&$\backslash$omit\&\&$\backslash$cr}
\bigskip\par\noindent
is designed to put a little space between the top horizontal line
and the title of the table. Although no text is specified for the first two 
columns of the table, \TeX\ does still obey the format specified for those two
columns. Hence, this line of the table is indented a half 
inch, has a vertical rule, and is the height of a normal 
line (remember the 
{\twltt $\backslash$strut}?). 
\par\noindent\bigskip
So what do we do to tell \TeX\ {\twlbf not} to insert any default 
``stuff'' from the template into a particular column? Enter 
{\twltt $\backslash$omit}\ \ We specify 
{\twltt $\backslash$omit} for our third column of this line so 
that we don't get an unwanted vertical rule 
dropping right in the middle of our column-spanning table 
title. 
\bigskip\par\noindent
The rest of this line of table-building commands operates in a 
similar fashion.
\bigskip\par\noindent
The next two lines, which looks like:
{\twltt \bigskip\par\noindent\hglue 0.5truein
$\backslash$multispan$\{$9$\}${}$\backslash$hglue 0.5in$\backslash$vrule$\backslash$hfill$\backslash$bf SOME SERVICE PISTOLS
\par\noindent\hglue 0.5truein
$\backslash$hfill$\backslash$vrule$\backslash$cr}
\bigskip\par\noindent
should be easy for you to decode now that we've explained the earlier 
table building commands -- there really isn't anything new here which you
haven't seen before, except that you should
notice that the commands to create a single line of table
output can be freely continued across multiple lines; \TeX\ knows a line of
table matter is finished only if and when it sees a {\twltt $\backslash$cr}
command.
\bigskip\par\noindent
In fact, you now know enough to decipher {\twlbf all} the rest of the 
table construction commands used to make our sample table.
\vfill\eject
%%%%%%%%%%%%%%%%%%%%%%%%%%%%%%%%%%%%%%%%%%%%%%%%%%%%%%%%%%%%%%%%%%%%%%%%%%%%%%%
\leftline{\twlbf ^{Table Body}}
\bigskip\par\noindent
One point which is easy to overlook: while you may bleed, sweat, and cry
while setting up a table's template and titles, the actual 
body of most tables is a {\twlit snap} if you set the template up correctly.
Thus, because so much of the table is handled automatically for you by
the template, the bigger a table is, the easier it is
to set it {\twlit once you get past} the template and title
material. For example, notice how easy it is to set 
the main part of our sample table:
\bigskip\par\noindent\hglue 0.5truein
{\twltt
\&\&\&\&\&\&\&\&$\backslash$cr
\par\noindent\hglue 0.5truein
\&Glock\&\&17\&\&9mm Para\&\&19\&$\backslash$cr
\par\noindent\hglue 0.5truein
\&Sig-Sauer\&\&P226\&\&9mm Para\&\&15\&$\backslash$cr
\par\noindent\hglue 0.5truein
\&Smith$\backslash$\&Wesson\&\&5926\&\&9mm Para\&\&15\&$\backslash$cr
\par\noindent\hglue 0.5truein
\&\&\&\&\&\&\&\&$\backslash$cr
\par\noindent\hglue 0.5truein
\&Glock\&\&21\&\&.45 ACP\&\&13\&$\backslash$cr
\par\noindent\hglue 0.5truein
\&Sig-Sauer\&\&P220\&\&.45 ACP\&\&7\&$\backslash$cr
\par\noindent\hglue 0.5truein
\&Smith$\backslash$\&Wesson\&\&4506\&\&.45 ACP\&\&8\&$\backslash$cr
\par\noindent\hglue 0.5truein
\&\&\&\&\&\&\&\&$\backslash$cr
}
\bigskip\par\noindent
Pretty straightforward, isn't it? Notice that you can even include actual
ampersands as text in your table by entering {\twltt $\backslash$\&}
for each ``real'' {\twltt \&}. 
\bigskip\par\noindent
As you become proficient at \TeX{}, you'll find that typesetting tables
isn't as difficult as you might have thought it was!
\vfill\eject
%%%%%%%%%%%%%%%%%%%%%%%%%%%%%%%%%%%%%%%%%%%%%%%%%%%%%%%%%%%%%%%%%%%%%%%%%%%%%%%
\centerline{\twlbf IV. ^{TYPESETTING EQUATIONS}}
\bigskip\bigskip\par\noindent
\centerline{\twlbf Typesetting Equations is Different From Typesetting Text}
\bigskip\par\noindent
^^{equation spacing}
Equations generally require special handling to come out looking ``right'' 
when printed. For example, mathematical variables are generally set in
italics (rather than roman fonts), operators (i.e., symbols such as plus
signs, minus signs, etc.) require special spacing, and a whole host of
other conventions need to be observed if the user is to end up with 
normal-looking mathematical copy.
\bigskip\par\noindent
Fortunately, \TeX\ is pre-programmed to automatically ``do the 
right thing'' when setting mathematical copy: you don't need to struggle
to get sweet-looking equations in \TeX. 
\bigskip\par\noindent
The remainder of this section will explain to you how to set the majority
of the mathematical copy an average \TeX\ user will need to 
set.\footnote{$^{10}$}{If you
are setting particularly esoteric or complicated equations, you'll probably
also want to closely study Chapters 16--19 of {\twlit The \TeX book} for
more detailed information about setting mathematical copy.}
\bigskip\par\noindent
We'll now consider the general context of how \TeX\ senses we're
setting mathematical copy, and then we'll
``zoom in'' to look at how we can generate various mathematical
symbols.
\bigskip\bigskip\par\noindent
\centerline{\twlbf ^{Embedded vs. Displayed Equations}}
^^{Displayed Equations}
\bigskip\par\noindent
Equations can be constructed two different ways in \TeX{}. Short, simple
equations are typically embedded right in the flow of narrative text, while
larger, more complex equations are usually indented and set as 
``displayed'' equations, separated from any adjacent text.
\bigskip\par\noindent
To signal \TeX\ that we're beginning to set an embedded equation,
begin the equation with a single dollar sign.
To signal the end of the embedded equation, use another dollar sign. For
example:
\bigskip\par\noindent\hglue 0.5truein
{\twltt Pythagoras' triangular equality \$c\^{}2=a\^{}2+b\^{}2\$ 
allows us to \${}$\backslash$ldots\$}
\bigskip\par\noindent
When processed by \TeX{}, your output would then look like:
\bigskip\par\noindent\hglue 0.5truein
Pythagoras' triangular equality $c^2=a^2+b^2$ allows us to $\dots$
\bigskip\par\noindent
When you want to set a displayed equation, begin the equation
with {\twlbf two} successive dollar signs
(instead of only one), and be sure to also end the  
equation with two dollar signs:
\bigskip\par\noindent\hglue 0.5truein
{\twltt Working the equation through, we arrive at the Sturm-Liouville}
\par\noindent\hglue 0.5truein
{\twltt expression:}
\par\noindent\hglue 0.5truein
{\twltt $\backslash$medskip$\backslash$par$\backslash$noindent}
{\twltt \$\$[r(x)y']+[q(x)+$\backslash$lambda\ p(x)]y=0\$\$}
\par\noindent\hglue 0.5truein
{\twltt Bessel, Legendre, and other equations can all be written in this form.}
\vfill\eject
%%%%%%%%%%%%%%%%%%%%%%%%%%%%%%%%%%%%%%%%%%%%%%%%%%%%%%%%%%%%%%%%%%%%%%%%%%%
\par\noindent
Processed, that looks like:
\bigskip\par\noindent\hglue 0.5truein
Working that equation through, we arrive at the Sturm-Liouville expression:
\medskip\par\noindent
$$[r(x)y']+[q(x)+\lambda p(x)]y=0$$
\par\noindent\hglue 0.5truein
Bessel, Legendre, and other equations can all be written in this form.
\bigskip\bigskip\par\noindent
\centerline{\twlbf ^{Numbering Equations}}
\bigskip\par\noindent
If you are preparing a complicated document with many equations, you will
almost certainly want to number your equations so that you can easily and
unambiguously refer to a particular equation in your narrative discussion.
\bigskip\par\noindent
The way to do that in \TeX\ is to build your displayed equation the way you
normally would, but then add:
\bigskip\par\noindent\hglue 0.5truein
{\twltt $\backslash$eqno({\twlit equationnumber})}
\bigskip\par\noindent
just before you enter the closing double dollar signs. For example:
\bigskip\par\noindent\hglue 0.5truein
{\twltt Working the equation through, we arrive at the }
\par\noindent\hglue 0.5truein
{\twltt Sturm-Liouville expression:}
\par\noindent\hglue 0.5truein
{\twltt $\backslash$medskip$\backslash$par$\backslash$noindent}
\par\noindent\hglue 0.5truein
{\twltt \$\$[r(x)y']+[q(x)+$\backslash$lambda\ p(x)]y=0$\backslash$eqno(8)\$\$}
\par\noindent\hglue 0.5truein
{\twltt Bessel, Legendre, and other equations can all be written in this form.}
\bigskip\par\noindent
Processed, the numbered version of our sample display equation looks like:
\bigskip\par\noindent\hglue 0.5truein
Working that equation through, we arrive at the Sturm-Liouville expression:
\medskip\par\noindent
$$[r(x)y']+[q(x)+\lambda p(x)]y=0\eqno(8)$$
\par\noindent\hglue 0.5truein
Bessel, Legendre, and other equations can all be written in this form.
\bigskip\bigskip\par\noindent
If you'd rather have your equation numbers on the left, instead of on the
right, substitute {\twltt $\backslash$leqno} for {\twltt $\backslash$eqno}.
Even if you want left hand equation numbers, you should still
insert the {\twltt $\backslash$leqno} immediately before the two dollar
signs {\twlbf ending} your displayed equation. Do {\twlbf not} move the {\twltt
$\backslash$leqno} so that it is just after the {\twlbf opening} double dollar
signs, or you'll generate an error.
\bigskip\par\noindent
There's another way to number equations, and that is as
part of aligning multiple equations in a single display. Just
coincidentally, that's the topic of our next section.
\vfill\eject
%%%%%%%%%%%%%%%%%%%%%%%%%%%%%%%%%%%%%%%%%%%%%%%%%%%%%%%%%%%%%%%%%%%%%%%%%%%%%%%
\centerline{\twlbf ^{Aligning and Numbering Multiple Equations}}
\bigskip\par\noindent
When you are preparing a particular mathematical display 
which contains more than one equation, you will usually want 
to align those equations by their 
equal signs (or some other reasonable visual ``balance'' point).
\bigskip\par\noindent
When you are doing that sort of ^{multiple equation alignment}, there
are two ways you can number your equations. You can number the block
of equations as a unit, or you can number all (or some) of the
equations in the block individually. 
\bigskip\par\noindent
%%%%%%%%%%%%%%%%%%%%%%%%%%%%%%%%%%%%%%%%%%%%%%%%%%%%
$\bullet$
^^{Numbering a block of equations}
If you want to number a block of equations {\twlbf as a unit}, you'd 
say something like:
\bigskip\par\noindent\hglue 0.5truein
{\twltt Consider the functions:}
\par\noindent\hglue 0.5truein
{\twltt \$\$$\backslash$eqalign$\{${}$\backslash$phi\_1=$\{${}$\backslash$rm arg$\}$(w+1)\&=$\backslash$arctan (v / [u+1])$\backslash$cr}
\par\noindent\hglue 0.5truein
\qquad\qquad\quad\enspace\thinspace{\twltt $\backslash$phi\_2=$\{$%
$\backslash$rm arg$\}$(w-1)\&=$\backslash$arctan
(v / [u-1])$\backslash$cr%
$\}$%
\par\noindent\hglue 0.5truein
\quad$\backslash$eqno(21)\$\$}
\par\noindent\hglue 0.5truein
{\twltt $\backslash$medskip$\backslash$par$\backslash$noindent}
\par\noindent\hglue 0.5truein
{\twltt We wish to solve these equations by \${}$\backslash$ldots\$}
\bigskip\par\noindent
Processed, that looks like:
\bigskip\par\noindent\hglue 0.5truein
Consider the functions:
\par\noindent\hglue 0.5truein
$$\eqalign{\phi_1={\rm arg}(w+1)&=\arctan (v / [u+1])\cr
           \phi_2={\rm arg}(w-1)&=\arctan (v / [u-1])\cr}\eqno(21)$$
\medskip\par\noindent\hglue 0.5truein
We wish to solve these equations by $\ldots$
\bigskip\bigskip\par\noindent
%%%%%%%%%%%%%%%%%%%%%%%%%%%%%%%%%%%%%%%%%%%%%%%%%%%%%%%%%%%%%%%%%%%%%%%%%%%%%%% 
$\bullet$ If you wanted to number {\twlbf every equation}
in the display you just built, you'd say, instead:
\bigskip\par\noindent\hglue 0.5truein
{\twltt Consider the functions:}
\par\noindent\hglue 0.5truein
{\twltt \$\$$\backslash$eqalignno$\{${}$\backslash$phi\_1=$\{${}$\backslash$rm arg$\}$(w+1)\&=$\backslash$arctan (v / [u+1])\&(45)$\backslash$cr}
\par\noindent\hglue 0.5truein
\qquad\qquad\quad\enspace\thinspace{\twltt \ \ $\backslash$phi\_2=$\{$%
$\backslash$rm arg$\}$(w-1)\&=$\backslash$arctan
(v / [u-1])\&(46)$\backslash$cr%
$\}$}
\par\noindent\hglue 0.5truein
{\twltt $\backslash$medskip$\backslash$par$\backslash$noindent}
\par\noindent\hglue 0.5truein
{\twltt We wish to solve these equations by \${}$\backslash$ldots\$}
\bigskip\par\noindent
Processed, the separately numbered {\twltt $\backslash$eqalignno} approach
generates output which looks like:
\bigskip\par\noindent\hglue 0.5truein
Consider the functions:
\par\noindent\hglue 0.5truein
$$\eqalignno{\phi_1={\rm arg}(w+1)&=\arctan (v / [u+1])&(45)\cr
           \phi_2={\rm arg}(w-1)&=\arctan (v / [u-1])&(46)\cr}$$
\bigskip\par\noindent\hglue 0.5truein
We wish to solve these equations by $\ldots$
\bigskip\par\noindent
Notice that this second approach uses {\twltt $\backslash$eqalign{\twlbf 
\underbar{no}}}
instead of simply {\twltt $\backslash$eqalign}, and notice that our equation
numbers are included immediately before the {\twltt $\backslash$cr} which ends
each line, separated from the actual equation itself by an ampersand ({\twltt
\&}).
\vfill\eject
%%%%%%%%%%%%%%%%%%%%%%%%%%%%%%%%%%%%%%%%%%%%%%%%%%%%%%%%%%%%%%%%%%%%%%%%%%%%%%%
\centerline{\twlbf Some ^{Basic Information About Entering Equations}}
\bigskip\par\noindent
But, you say, it is all fine and good to be able to know the
difference between embedded and displayed equations, and to know
how to number and align equations,
but how do I actually write the damn things in the first place!{}?
Ah! Yes! I guess we should talk about that, eh?
\bigskip\par\noindent
Regardless of whether you are entering an embedded equation, or an equation
which will be displayed, there are some basic things you need to know about
entering equations in \TeX:
\bigskip\par\noindent
$\bullet$ \TeX\ generally ignores spaces
in equations. Thus, the following are equivalent to \TeX\ and will
come out typeset exactly the same:
\bigskip\par\noindent\hglue 0.5truein
{\twltt \$\$ a\ +\ b\ +\ c\ \ =\ \ d\ \$\$}
\bigskip\par\noindent\hglue 0.5truein
{\twltt \$\$a+b+c=d\$\$}
\bigskip\par\noindent
$\bullet$ To force \TeX\ to pay attention to spaces
in your equations, use the following:
\bigskip\par\noindent\hglue 0.5truein
{\twltt $\backslash$! \qquad ^{negative thin space} (-1/6 $\backslash$quad)}
\par\noindent\hglue 0.5truein
{\twltt $\backslash$, \qquad\ ^{thin space} (+1/6 $\backslash$quad)}
\par\noindent\hglue 0.5truein
{\twltt $\backslash$> \qquad\ ^{medium space} (+2/9 $\backslash$quad)}
\par\noindent\hglue 0.5truein
{\twltt $\backslash$; \qquad\ ^{thick space} (+5/18 $\backslash$quad)}
\par\noindent\hglue 0.5truein
{\twltt $\backslash$quad \ \ one ^{quad} -- approximately the size of a 
capital "M"}
\par\noindent\hglue 0.5truein
{\twltt $\backslash$qquad \ a ^{double quad}}
\par\noindent\hglue 0.5truein
{\twltt $\backslash$\char `\ \qquad\ \ a ^{control space}
\bigskip\par\noindent}
$\bullet$ In general, math mode text is set in 
italics. To force \TeX\ to set a few words of text in a normal 
roman font while in math mode, you can put the text in a 
^^{roman font words in math mode}
{\twltt $\backslash$rm} group:
\bigskip\par\noindent\hglue 0.5truein
{\twltt \$\$22/7 
$\{${}$\backslash$rm$\backslash$\ is $\backslash$ an  
$\backslash$\ approximation $\backslash$\ for $\backslash$\ $\}$
$\backslash$pi\$\$}
\bigskip\par\noindent
which yields output that looks like:
\bigskip\par\noindent
$$22/7 {\rm \ is \ an \ approximation \ for \ } \pi$$
\bigskip\par\noindent
But notice what a pain that is to set! 
\bigskip\par\noindent
By far your best move is to avoid setting {\twlbf displayed}
equations which include lots of ``regular'' text. 
Use {\twlbf embedded} equations instead, and just drop
into math mode when you need to on a ``phrase-by-phrase'' basis. For
example, look how much cleaner the following re-casting of our example
becomes:
\bigskip\par\noindent\hglue 0.5truein
{\twltt \$22/7\$ is an approximation for \${}$\backslash$pi\$}
\bigskip\par\noindent
Processed, that looks like:
\bigskip\par\noindent\hglue 0.5truein
$22/7$ is an approximation for $\pi$
\bigskip\par\noindent
\TeX\ also contains pre-defined roman font text strings for common
mathematical ``words'' such as sin, cos, tan, lim, etc. See below
for more information.
\vfill\eject
%%%%%%%%%%%%%%%%%%%%%%%%%%%%%%%%%%%%%%%%%%%%%%%%%%%%%%%%%%%%%%%%%%%%%%%%%%%%%%%
\centerline{\twlbf ^{Greek Letters}}
\bigskip\par\noindent
$\bullet$ To enter {\twlbf lowercase} greek letters 
in math mode (i.e., between \$ $\ldots$ \$ or \$\$ $\ldots$ \$\$), use:
\bigskip\par\noindent
\settabs 6\columns
%%%%%%%%%
\+$\alpha$&$\beta$&$\gamma$&$\delta$&$\epsilon$&$\varepsilon$\cr
\medskip\par\noindent
\+$\backslash$alpha&$\backslash$beta&
$\backslash$gamma&$\backslash$delta&
$\backslash$epsilon&$\backslash$varepsilon\cr
\bigskip\par\noindent
%%%%%%%%%
\+$\zeta$&$\eta$&$\theta$&$\vartheta$&$\iota$&$\kappa$\cr
\medskip\par\noindent
\+$\backslash$zeta&
$\backslash$eta&$\backslash$theta&$\backslash$vartheta&
$\backslash$iota&$\backslash$kappa\cr
\bigskip\par\noindent
%%%%%%%%%%
\+$\lambda$&$\mu$&$\nu$&$\xi$&$o$&$\pi$\cr
\medskip\par\noindent
\+$\backslash$lambda&$\backslash$mu&$\backslash$nu&%
$\backslash$xi&o&$\backslash$pi\cr
\bigskip\par\noindent
%%%%%%%%%%
\+$\varpi$&$\rho$&$\varrho$&$\sigma$&$\varsigma$&$\tau$\cr
\medskip\par\noindent
\+$\backslash$varpi&$\backslash$rho&$\backslash$varrho&%
$\backslash$sigma&$\backslash$varsigma&$\backslash$tau\cr
\bigskip\par\noindent
%%%%%%%%%%
\+$\upsilon$&$\phi$&$\varphi$&$\chi$&$\psi$&$\omega$\cr
\medskip\par\noindent
\+$\backslash$upsilon&$\backslash$phi&$\backslash$varphi&
$\backslash$chi&$\backslash$psi&$\backslash$omega\cr
\bigskip\bigskip\par\noindent
%%%%%%%%%%%
$\bullet$ To enter {\twlbf uppercase} greek letters 
in math mode (i.e., between \$ $\ldots$ \$ or \$\$ $\ldots$ \$\$), use:
\bigskip\par\noindent
\settabs 6\columns
\+$\Gamma$&$\Delta$&$\Theta$&$\Lambda$&$\Xi$&$\Pi$\cr
\medskip\par\noindent
\+$\backslash$Gamma&$\backslash$Delta&
$\backslash$Theta&$\backslash$Lambda&$\backslash$Xi&$\backslash$Pi\cr
\bigskip\par\noindent
%%%%%%%%%%%%
\+$\Sigma$&$\Upsilon$&$\Phi$&$\Psi$&$\Omega$\cr
\medskip\par\noindent
\+$\backslash$Sigma&$\backslash$Upsilon&
$\backslash$Phi&$\backslash$Psi&$\backslash$Omega\cr
\bigskip\par\noindent
Only those uppercase greek letters shown above are available in \TeX{};
those not defined are indistinguishable from their roman counterparts.
Thus, for example, {\twltt $\backslash$Alpha} is undefined, and will
generate an error if you attempt to include that undefined symbol in
your document. Simply substitute {\twltt $\{${}$\backslash$rm A$\}$}
instead at the point where you need a capital alpha. For italic 
uppercase greek letters, try: {\twltt \${}$\{${}$\backslash$mit
$\backslash$Phi$\}$}\$.
\bigskip\bigskip\par\noindent
\centerline{\twlbf ^{Script Letters}}
\bigskip\par\noindent
Sometimes you'll need to enter a capital ``script'' letter while in
math mode (i.e., , between \$ $\ldots$ \$ or \$\$ $\ldots$ \$\$). Use
the calligraphic style for that purpose (lowercase isn't available):
\bigskip\par\noindent\hglue 0.5truein
\+&${\cal ABCDEFGHIJKLMNOPQRSTUVWXYZ}$\cr
\medskip\par\noindent\hglue 0.5truein
\+&$\{${}$\backslash$cal ABCDEFGHIJKLMNOPQRSTUVWXYZ$\}${}\cr
\vfill\eject
%%%%%%%%%%%%%%%%%%%%%%%%%%%%%%%%%%%%%%%%%%%%%%%%%%%%%%%%%%%%%%%%%%%%%%%%%%%%%%%
\centerline{\twlbf Common ^{Mathematical Operators}}
\bigskip\par\noindent
You also need to be able to set common mathematical symbols. For example,
while in math mode (i.e., , between \$ $\ldots$ \$ or \$\$ $\ldots$ \$\$) 
some of the symbols you can use include:
%%%%%%%%%%%%%%%%%%
\bigskip\par\noindent
\+$+$&$-$&$=$&$\not=$&$<$&$>$\cr
\medskip\par\noindent
{\twltt \+{}+&-&=&$\backslash$not=&<&>\cr}
%%%%%%%%%%%%%%%%%%%
\bigskip\bigskip\par\noindent
\+$\le$&$\ge$&$\ll$&$\gg$&$\approx$&$\equiv$\cr
\medskip\par\noindent
{\twltt \+$\backslash$le&$\backslash$ge&$<<$&$>>$&%
$\backslash$approx&$\backslash$equiv\cr}
%%%%%%%%%%%%%%%%%%%
\bigskip\bigskip\par\noindent
\+(&)&[&]&$\{$&$\}$\cr
\medskip\par\noindent
{\twltt \+(&)&[&]&$\backslash${}$\{$&$\backslash${}$\}$\cr}
%%%%%%%%%%%%%%%%%%%%
\bigskip\bigskip\par\noindent
\+$\langle$&$\rangle$&%
$\colon$&$\cdot$&$\cdots$&$\ldots$\cr
\medskip\par\noindent
{\twltt \+$\backslash$langle&$\backslash$rangle&%
$\backslash$colon&$\backslash$cdot&$\backslash$cdots&%
$\backslash$ldots\cr}
%%%%%%%%%%%%%%%%%%%%%
\bigskip\bigskip\par\noindent
\+$\vert$&$\Vert$&$\backslash$&$\lceil$&$\lfloor$&$\nabla$\cr
\medskip\par\noindent
{\twltt \+$\backslash$vert&$\backslash$Vert&$\backslash$backslash&%
$\backslash$lceil&$\backslash$lfloor&$\backslash$nabla\cr}
%%%%%%%%%%%%%%%%%%%%%
\bigskip\bigskip\par\noindent
\+$\partial$&$\infty$&$\forall$&$\exists$&$\circ$&$\bullet$\cr
\medskip\par\noindent
{\twltt \+$\backslash$partial&$\backslash$infty&$\backslash$forall&
$\backslash$exists&$\backslash$circ&$\backslash$bullet\cr}
%%%%%%%%%%%%%%%%%%%
\bigskip\bigskip\par\noindent
%%%%%%%%%%%%%%%%%%%%%%%%%%%%%%%%%%%%%%%%%%%%%%%%%%%%%%%%%%%%%%%%%%%%%%%%%%%%%%%
\centerline{\twlbf Symbols for ^{Logic} and the ^{Algebra of Sets}}
\bigskip\par\noindent
\+$\in$&$\notin$&$\cap$&$\cup$&$\vee$&$\wedge$\cr
\medskip\par\noindent
{\twltt \+$\backslash$in&$\backslash$notin&$\backslash$cap&%
$\backslash$cup&$\backslash$vee&$\backslash$wedge\cr}
\bigskip\bigskip\par\noindent
\+$\subset$&$\supset$&$\subseteq$&$\supseteq$&$\oplus$&$\otimes$\cr
\medskip\par\noindent
{\twltt \+$\backslash$subset&$\backslash$supset&
$\backslash$subseteq&$\backslash$supseteq&%
$\backslash$oplus&$\backslash$otimes\cr}
\bigskip\bigskip\par\noindent
\+$\sim$&$\ast$&$\emptyset$&$\rightarrow$&$\leftarrow$&$\leftrightarrow$\cr
\medskip\par\noindent
{\twltt \+$\backslash$sim&$\backslash$ast&$\backslash$emptyset&%
$\backslash$rightarrow&$\backslash$leftarrow&%
$\backslash$leftrightarrow\cr}
\bigskip\bigskip\par\noindent
For more information on available mathematical symbols, see Appendix F of
{\twlit The \TeX{}book}.
\vfill\eject
%%%%%%%%%%%%%%%%%%%%%%%%%%%%%%%%%%%%%%%%%%%%%%%%%%%%%%%%%%%%%%%%%%%%%%%%%%%%%%
\centerline{\twlbf ^{Subscripts}, ^{Superscripts}, and Combinations Thereof}
\bigskip\par\noindent
While in math mode, make subscripts using an underscore and 
make superscripts using a ^{caret} (or ``^{hat}''). For example:
\bigskip\par\noindent
\+Subscript: &&&
{\twltt \$X\_{}$\{$14$\}${}\$}&& $X_{14}$\cr
\bigskip\par\noindent
\+Superscript: &&&
{\twltt \$X\^{}$\{$3$\}${}\$}&& $X^{3}$\cr
\bigskip\par\noindent
\+Combined subscript and superscript: &&&
{\twltt \$X\_{}$\{$6$\}${}\^{}$\{$8$\}${}\$}&& $X_{6}^{8}$\cr
\bigskip\par\noindent
\+Even odder combinations, too: &&&
{\twltt \${}\_$\{$4$\}${}X\^{}$\{$a\^{}b$\}$}&&$_{4}X^{a^b}$\cr
\bigskip\par\noindent
\+&&&
{\twltt \${}\^{}$\{$2$\}${}\_$\{$3$\}${}X\_{}$\{$5$\}${}\^{}$\{$1$\}${}\$}%
&&$^{2}_{3}X_{5}^{1}$\cr
\bigskip\bigskip\par\noindent
Note that you may use (and you may {\twlit need to} use) curly braces
to indicate exactly what material is to be raised or lowered; if a particularly
baroque expression is ambiguous, \TeX\ will complain.
\bigskip\par\noindent
Also note that you can override the automatic reduction in ^{size of
subscripts and superscripts}. Do so by including a 
{\twltt $\backslash$textstyle}
within the subscript or superscript curly braces:
\bigskip\par\noindent
\+&{\twltt \$X\^{}%
$\{${}%
$\{${}$\backslash$textstyle Y$\}${}\^{}%
$\{${}$\backslash$textstyle Z$\}${}$\}$%
{}\$}%
&&&&$X^{{\textstyle Y}^{\textstyle Z}}$\cr
\bigskip\bigskip\par\noindent
%%%%%%%%%%%%%%%%%%%%%%%%%%%%%%%%%%%%%%%%%%%%%%%%%%%%%%%%%%%%%%%%%%%%%%%%%%%%%%%
\centerline{\twlbf ^{Math Accents}}
\bigskip\par\noindent
While in math mode, get accented variables\footnote{$^{11}$}{Note that
mathematical typesetting conventions require you to use a special
``dotless'' i and ``dotless'' j when accenting those characters. 
^^{dotless characters}
Get those
characters by saying {\twltt $\backslash$imath} to get a dotless $\imath$ 
or by saying {\twltt $\backslash$jmath} to get a dotless $\jmath$ where 
needed in math mode.}
by saying:
\bigskip\par\noindent
\+$x'$&$x''$&$\bar x$&$\hat x$&$\vec x$&$\dot x$\cr
\medskip\par\noindent
{\twltt \+x{}'&x{}'{}'&$\backslash$bar x&$\backslash$hat x&
$\backslash$vec x&$\backslash$dot x\cr}
\bigskip\bigskip\par\noindent
\+$\ddot x$&$\check x$&$\breve x$&$\grave x$&$\acute x$&$\tilde x$\cr
{\twltt \+$\backslash$ddot x&$\backslash$check x&$\backslash$breve x&
$\backslash$grave x&$\backslash$acute x&$\backslash$tilde x\cr}
\bigskip\bigskip\par\noindent
\bigskip\bigskip\par\noindent
\settabs 4\columns
^^{Wide math accents}
``Wider'' accents are also available:
\bigskip\par\noindent
\+$\overline {abcde}$&$\underline {abcde}$&$\widehat {abcde}$&
$\widetilde {abcde}$\cr
\medskip\par\noindent
{\twltt \+$\backslash$overline $\{$abcde$\}$&
$\backslash$underline $\{$abcde$\}$&
$\backslash$widehat $\{$abcde$\}$&
$\backslash$widetilde $\{$abcde$\}$\cr}
\settabs 6\columns
\vfill\eject
%%%%%%%%%%%%%%%%%%%%%%%%%%%%%%%%%%%%%%%%%%%%%%%%%%%%%%%%%%%%%%%%%%%%%%%%%%%%%%%
\centerline{\twlbf ^{Roman Font Mathematical ``Words''}}
\bigskip\bigskip\par\noindent
\TeX\ includes a number of predefined roman font math-mode ``words'':
\bigskip\bigskip\par\noindent
$\bullet$ {\twlbf Trigonometric Functions:}
\bigskip\par\noindent
%%%%%%%%%%%%%%
\+$\sin$&$\cos$&$\tan$&$\csc$&$\sec$&$\cot$\cr
\medskip\par\noindent
{\twltt \+$\backslash$sin&$\backslash$cos&$\backslash$tan&%
$\backslash$csc&$\backslash$sec&$\backslash$cot\cr}
\bigskip\par\noindent
%%%%%%%%%%%%%%%
\+$\arcsin$&$\arccos$&$\arctan$&$\sinh$&$\cosh$&$\tanh$\cr
\medskip\par\noindent
{\twltt \+$\backslash$arcsin&$\backslash$arccos&$\backslash$arctan&
$\backslash$sinh&$\backslash$cosh&%
$\backslash$tanh&\cr}
\bigskip\par\noindent
%%%%%%%%%%%%%%%
\+$\coth$\cr
\medskip\par\noindent
{\twltt \+$\backslash$coth\cr}
\bigskip\bigskip\par\noindent
%%%%%%%%%%%%%%%%
Those ^{trigonometric functions} not shown in the above list (including
the remaining inverse trigonometric functions, the remaining hyperbolic
functions, the inverse hyperbolics, exsec, covers, vers, hav, cis, etc.),
need to be manually entered as roman text while you're in math
mode since \TeX\ considers them too obscure to merit pre-defined
symbols.
\bigskip\bigskip\par\noindent
$\bullet${\twlbf Other Functions:}
\bigskip\par\noindent
The following list of other functions is something of a hodge-podge,
consisting of the other roman font mathematical ``words'' predefined in
plain \TeX{}:
%%%%%%%%%%%%%%%%%
\bigskip\par\noindent
\+$\min$&$\max$&$\ln$&$\log$&$\lg$&$\exp$\cr
\medskip\par\noindent
{\twltt \+$\backslash$min&$\backslash$max&$\backslash$ln&%
$\backslash$log&$\backslash$lg&$\backslash$exp\cr}
\bigskip\par\noindent
%%%%%%%%%%%%%%%%%%
\+$\gcd$&$\inf$&$\sup$&$\lim$&$\liminf$&$\limsup$\cr
\medskip\par\noindent
{\twltt \+$\backslash$gcd&$\backslash$inf&$\backslash$sup&$\backslash$lim&%
$\backslash$liminf&$\backslash$limsup\cr}
\bigskip\par\noindent
%%%%%%%%%%%%%%%%%%%
\+$\dim$&$\det$&$\arg$&$\deg$&$\hom$&$\ker$&\cr
\medskip\par\noindent
{\twltt \+$\backslash$dim&$\backslash$det&$\backslash$arg&%
$\backslash$deg&$\backslash$hom&$\backslash$ker\cr}
\bigskip\par\noindent
%%%%%%%%%%%%%%%%%%%%
\+$\Pr$&$\bmod$&$\!\!\!\!\!\!\pmod{16}$\cr
\medskip\par\noindent
{\twltt \+$\backslash$Pr&$\backslash$bmod&$\backslash$pmod$\{$16$\}$\cr}
\bigskip\bigskip\par\noindent
$\bullet$
If the mathematical ``word'' you find yourself needing isn't listed, you can
always extend the above list to meet your circumstances. For example:
\bigskip\bigskip\par\noindent\hglue 0.5truein
{\twltt $\backslash$def$\backslash$lcd$\{${}$\backslash$mathop$\{${}%
$\backslash$rm lcd$\}${}$\backslash$nolimits$\}$}
\bigskip\par\noindent
^^{Making New Roman Font Math Symbols}
would define a new symbol {\twltt $\backslash$lcd} as a convenience for those
who commonly write about least common denominators.
\vfill\eject
%%%%%%%%%%%%%%%%%%%%%%%%%%%%%%%%%%%%%%%%%%%%%%%%%%%%%%%%%%%%%%%%%%%%%%%%%%%%%%
\centerline{\twlbf ^{Limits}}
\bigskip\par\noindent
One example of using a ``named'' mathematical function is writing a limit:
\bigskip\par\noindent\hglue 0.5truein
{\twltt \$\${}$\{${}$\backslash$bf u$\}${}'(t)=$\backslash$lim\_$\{${}%
$\backslash$Delta t$\backslash$rightarrow 0$\}$}
\par\noindent\hglue 0.5truein
{\twltt \qquad$\{\{\{\backslash$bf u$\}$ (t+$\backslash$Delta t) 
- $\{\backslash$bf u$\}$(t)$\}$ 
$\backslash$over $\{${}$\backslash$Delta 
t$\}\}${}\$\$}
\bigskip\par\noindent
You'll get output from those commands that looks like:
\bigskip\par\noindent
$${\bf u}'(t)=\lim_{\Delta t\rightarrow 0} 
{{{\bf u} (t+\Delta t) - {\bf u}(t)} \over {\Delta t}}$$
\bigskip\par\noindent
Note that the ``named'' mathematical function can have a subscript just
the way any ``regular'' symbol can, but that \TeX\ is smart enough to 
center the limit under this particular
named function rather than simply dropping the subscript below the base 
line at the end of the symbol. This decision
--- to drop the subscript down in place, or to drop the subscript
down and then pull it back to center it --- is controlled by the {\twltt 
$\backslash$limits} or {\twltt $\backslash$nolimits} attribute chosen 
when that symbol was defined. (You may have noticed that little feature 
in our ``{\twltt $\backslash$lcd}'' example in the preceding section. Now
you know $\ldots$ the {\twlit rest} of the story!)
\bigskip\bigskip\par\noindent
\centerline{\twlbf ^{Radicals}: ^{Square Roots}, ^{Cube Roots}, etc.}
\bigskip\par\noindent
$\bullet$
To get a square root as part of a math mode expression, try something like:
\bigskip\par\noindent\hglue 0.5truein
{\twltt $\backslash$sqrt$\{$x+y$\}$} \qquad to get: \qquad
$\sqrt{x+y}$
\bigskip\par\noindent
$\bullet$ \TeX{}'s {\twltt $\backslash$sqrt} will attempt to be ``smart'' 
and match the radical height to the height of the arguments given:
{\twltt $\backslash$sqrt} will make a bigger radical 
for ``tall'' arguments (such as those containing capital letters) 
than it would for arguments comprised solely of ``short'' 
lower-case letters. 
\bigskip\par\noindent
However, there may be times when you {\twlit don't want} \TeX\ to 
tailor each {\twltt $\backslash$sqrt} radical individually; at times
it may be cleaner to have all the {\twltt $\backslash$sqrt} radicals
be of the same height. When that's the case, include a {\twltt 
$\backslash$mathstrut} as part of each radical in your equation:
\bigskip\par\noindent\hglue 0.5truein
{\twltt $\backslash$sqrt$\{${}$\backslash$mathstrut x+y$\}$ +
        $\backslash$sqrt$\{${}$\backslash$mathstrut Z$\}$}
\bigskip\par\noindent
All of your radicals will then be the same height:
\bigskip\par\noindent\hglue 0.5truein
$\sqrt{\mathstrut x+y} + \sqrt{\mathstrut Z}$
\bigskip\bigskip\par\noindent
$\bullet$ To get any arbitrary root (cube root, quartic root, etc.), 
while in math mode try:
\bigskip\par\noindent\hglue 0.5truein
{\twltt $\backslash$root 4 $\backslash$of X}
\bigskip\par\noindent
You'll then get output which looks like:
\bigskip\par\noindent\hglue 0.5truein
$\root 4 \of X$
\vfill\eject
%%%%%%%%%%%%%%%%%%%%%%%%%%%%%%%%%%%%%%%%%%%%%%%%%%%%%%%%%%%%%%%%%%%%%%%%%%%%%%%
\centerline{\twlbf ^{Making Large Fractions}}
^^{Large Fractions}
^^{Fractions}
\bigskip\par\noindent
To generate a ``large'' fraction while in math mode, use the 
{\twltt $\backslash$over} command:
\bigskip\par\noindent\hglue 0.5truein
\$\${}$\{$x+y $\backslash$over a+b$\}${}\$\$
\bigskip\par\noindent
That generates output which looks like:
\bigskip\par\noindent\hglue 0.5truein
$${x+y \over a+b}$$
\bigskip\bigskip\par\noindent
$\bullet$ If you need to generate ``fractions upon fractions'', you can quickly
end up with something which looks like:
\bigskip\par\noindent\hglue 0.5truein
{\twltt \$\${}$\{${}$\{$x+y $\backslash$over a+b$\}${} $\backslash$over
$\{$x+a $\backslash$over y+b$\}${}$\}${}\$\$}
\bigskip\par\noindent
and, even worse, the output from that ``fraction upon fraction'' form
tends to be hard to read, ambiguous and wasteful of space in the printed page:
$${{x+y \over a+b} \over {x+a \over y+b}}$$
\medskip\bigskip\par\noindent
$\bullet$ To disambiguate the ``fraction upon fraction'' form, you can employ a
bold division line:
\bigskip\par\noindent\hglue 0.5truein
{\twltt \$\${}$\{${}$\{$x+y $\backslash$over a+b$\}${} $\backslash$above 1pt
$\{$x+a $\backslash$over y+b$\}${}$\}${}\$\$}
\bigskip\par\noindent
That yields the following non-ambiguous expression:
\bigskip\par\noindent\hglue 0.5truein
$${{x+y \over a+b}\above 1pt {x+a \over y+b}}$$
\medskip\par\noindent
Even though that ``fraction upon fraction'' format is now unambiguous, it is
still hard to read and still is wasteful of space on the printed page.
\bigskip\bigskip\par\noindent
$\bullet$ You are far better off writing ``fractions upon fractions'' 
horizontally, with an enlarged slash. See the following section on 
enlarged grouping operators for more information.
\vfill\eject
%%%%%%%%%%%%%%%%%%%%%%%%%%%%%%%%%%%%%%%%%%%%%%%%%%%%%%%%%%%%%%%%%%%%%%%%%%%%%%%
\centerline{\twlbf ^{Making Large Grouping Operators}}
\bigskip\par\noindent
When you begin working with large and complicated displayed equations, a
number of ^{large grouping operators} will become important to you. 
These are:
\bigskip\par\noindent\hglue 0.5truein
$\bullet$ Large ^{parentheses}, ^{braces}, and ^{brackets},
\par\noindent\hglue 0.5truein
$\bullet$ Large vertical (``absolute value'') bars, and
^^{absolute value bars}
^^{vertical bars} 
\par\noindent\hglue 0.5truein
$\bullet$ Large forward and backward slashes.
^^{forward slashes}
^^{backward slashes}
\bigskip\bigskip\par\noindent
Five different sizes of those operators can be requested:
\bigskip\par\noindent\hglue 0.5truein
$\bullet$ Normal size
\par\noindent\hglue 0.5truein
$\bullet$ {\twltt $\backslash$big}, {\twltt $\backslash$bigl} 
and {\twltt $\backslash$bigr} --- a hair bigger than normal
\par\noindent\hglue 0.5truein
$\bullet$ {\twltt $\backslash$Big}, {\twltt $\backslash$Bigl} 
and {\twltt $\backslash$Bigr} --- 1.5 times the {\twltt 
$\backslash$big} size operators
\par\noindent\hglue 0.5truein
$\bullet$ {\twltt $\backslash$bigg}, {\twltt $\backslash$biggl} 
and {\twltt $\backslash$biggr} ---
2.0 times the {\twltt $\backslash$big} size operators
\par\noindent\hglue 0.5truein
$\bullet$ {\twltt $\backslash$Bigg}, {\twltt $\backslash$Biggl} 
and {\twltt $\backslash$Biggr} ---
2.5 times the {\twltt $\backslash$big} size operators
\bigskip\par\noindent
Thus, for example, you might write something like:
\bigskip\par\noindent\hglue 0.5truein
\$\$w=$\backslash$biggl($\{$a+b $\backslash$over d+e$\}${}$\backslash$biggr) 
$\backslash$bigg/ $\backslash$biggl($\{$f+g+h $\backslash$over i+j$\}${}$\backslash$biggr)\$\$
\bigskip\par\noindent
In order to obtain output that looks like:
\bigskip\par\noindent
$$w=\biggl({a+b \over d+e}\biggr) \bigg/ \biggl({f+g+h \over i+j}\biggr)$$
\bigskip\par\noindent
You'll notice that I used {\twltt $\backslash$bigg\underbar{l}(} for the 
left hand side parentheses, {\twltt $\backslash$bigg\underbar{r})} for the 
right hand side parentheses, and plain {\twltt $\backslash$bigg} for the
center division sign. 
\TeX\ handles spacing around the big grouping operators differently 
depending on which ``side'' the operator is on.
Unless you have a compelling reason to the contrary, you should generally
use {\twltt $\backslash$biggl} for left hand side operators, {\twltt
$\backslash$biggr} for right hand side operators, and {\twltt $\backslash$big}
for those operators which cannot really be called either left hand side or 
right hand side.
\bigskip\par\noindent
To give you an idea of what the various symbols look like at the five
available sizes, here's a little sampler: 
\settabs 4\columns
\bigskip\bigskip\par\noindent
\+\hfill $\Biggl( \biggl( \Bigl( \bigl( \,(X)\, \bigr) \Bigr) \biggr) 
\Biggr)$ \hfill & \hfill
$\Biggl[ \biggl[ \Bigl[ \bigl[ \,[X]\, \bigr] \Bigr] \biggr] \Biggr]$ \hfill &
\hfill $\Biggl\{ \biggl\{ \Bigl\{ \bigl\{ \, \{X\} \, \bigr\} 
\Bigr\} \biggr\} \Biggr\} $ \hfill & 
\quad \hfill 
$\Biggl\langle \biggl\langle \Bigl\langle \bigl\langle \, 
\langle X \rangle \,
\bigr\rangle \Bigr\rangle \biggr\rangle \Biggr\rangle$ \hfill \cr
\bigskip\par\noindent
\+ \hfill $\Biggl\vert \biggl\vert \Bigl\vert \bigl\vert \,\vert X 
\vert\, \bigr\vert 
\Bigr\vert \biggr\vert \Biggr\vert$ \hfill & \hfill 
$\Biggl\Vert \biggl\Vert \Bigl\Vert \bigl\Vert \Vert X \Vert \bigr\Vert 
\Bigr\Vert \biggr\Vert \Biggr\Vert$ \hfill & \hfill
$\Biggl/ \biggl/ \Bigl/ \bigl/ / X / \bigr/ \Bigr/ \biggr/ \Biggr/ $ \hfill &
\hfill $\Biggl\backslash \biggl\backslash \Bigl\backslash \bigl\backslash 
\,\backslash X \backslash\, \bigr\backslash \Bigr\backslash \biggr\backslash 
\Biggr\backslash$ \hfill \cr
\vfill\eject
%%%%%%%%%%%%%%%%%%%%%%%%%%%%%%%%%%%%%%%%%%%%%%%%%%%%%%%%%%%%%%%%%%%%%%%%%%%%%%%
\centerline{\twlbf ^{Combination Notation}}
\bigskip\par\noindent
To denote combinations in \TeX{}, use \TeX{}'s {\twltt $\backslash$choose}
notation:
\bigskip\par\noindent\hglue 0.5truein
{\twltt \${}$\{$n $\backslash$choose r$\}$\$}
\qquad
will print as:
\qquad
{\twltt ${n \choose r}$}
\bigskip\bigskip\par\noindent
\centerline{\twlbf ^{Matrices}}
\bigskip\par\noindent
To typeset a matrix, enter something like:
\bigskip\par\noindent
{\twltt \$\${}$\{${}$\backslash$bf X$\}$ = 
$\backslash$left($\backslash$matrix$\{$}
\par\noindent
{\twltt \enspace\quad\qquad\qquad\qquad\qquad\qquad\qquad X\_$\{$11$\}$\&%
X\_$\{$12$\}$\&X\_$\{$13%
$\}$\&$\backslash$ldots\&X\_$\{$1j$\}${}$\backslash$cr}
\par\noindent
{\twltt \enspace\quad\qquad\qquad\qquad\qquad\qquad\qquad 
X\_$\{$21$\}$\&X\_$\{$22$\}$\&%
X\_$\{$23$\}$\&%
$\backslash$ldots\&X\_$\{$2j$\}${}$\backslash$cr}
\par\noindent
{\twltt \enspace\quad\qquad\qquad\qquad\qquad\qquad\qquad
 $\backslash$vdots\&$\backslash$vdots\&%
$\backslash$vdots\&$\backslash$ddots\&%
$\backslash$vdots$\backslash$cr}
\par\noindent
{\twltt \enspace\quad\qquad\qquad\qquad\qquad\qquad\qquad 
X\_$\{$i1$\}$\&X\_$\{$i2$\}$\&%
X\_$\{$i3$\}$\&%
$\backslash$ldots\&X\_$\{$ij$\}${}$\backslash$cr}
\par\noindent
{\twltt \qquad\qquad\qquad$\}${}$\backslash$right)$\backslash$eqno(4)\$\$}
\bigskip\par\noindent
Processed by \TeX{}, that looks like:
\bigskip\par\noindent
$$ {\bf X} = \left( \matrix{X_{11}&X_{12}&X_{13}&\ldots&X_{1j}\cr
                           X_{21}&X_{22}&X_{23}&\ldots&X_{2j}\cr
                           \vdots&\vdots&\vdots&\ddots&\vdots\cr
                           X_{i1}&X_{i2}&X_{i3}&\ldots&X_{ij}\cr} 
                    \right)\eqno(4)$$
\bigskip\par\noindent
Note that you can use symbols other than parentheses to set off your matrix;
common alternatives include large square brackets and large 
bars (for denoting determinants).
\bigskip\bigskip\par\noindent 
\centerline{\twlbf ^{Case Structure}}
\bigskip\par\noindent
Sometimes you need a ``case'' structure to represent choices and outcomes; 
for example, maybe you're trying to show quantities and price breaks:
\bigskip\par\noindent\hglue 0.5truein
{\twltt \$\${}$\{${}$\backslash$rm your$\backslash$ price$\}$ $=$}
\par\noindent\hglue 0.5truein
{\twltt \enspace\quad$\backslash$cases$\{${}$\backslash$\$10.00/$\{${}$\backslash$rm 
unit$\}$\& for 1,000 or more units$\backslash$cr}
\par\noindent\hglue 0.5truein
{\twltt \enspace\enspace
\qquad\qquad$\backslash${}\$12.50/$\{${}$\backslash$rm unit$\}$\& 
for 500 to 999
units$\backslash$cr}
\par\noindent\hglue 0.5truein
{\twltt \enspace\enspace\qquad\qquad$\backslash${}\$15.00/$\{${}$\backslash$rm
unit$\}$\& for 1 to 499
units$\backslash$cr$\}${}$\backslash$eqno(16)\$\$}
\bigskip\par\noindent
That produces output which looks like:
\par\noindent\hglue 0.5truein
$${\rm your\ price} = 
   \cases{\$10.00/{\rm unit}& for 1,000 or more units\cr
          \$12.50/{\rm unit}& for 500 to 999 units\cr
          \$15.00/{\rm unit}& for 1 to 499 units\cr}\eqno(16)$$
\bigskip\par\noindent
Note that the material to the left of the equal sign, and the material to the
left of the ampersand sign, is automatically typeset in math mode. 
The material to the right of the ampersand sign is automatically typeset
in non-math mode.
\vfill\eject
%%%%%%%%%%%%%%%%%%%%%%%%%%%%%%%%%%%%%%%%%%%%%%%%%%%%%%%%%%%%%%%%%%%%%%%%%%%%%%%
\centerline{\twlbf ^{Summations}}
^^{sigmas}
\bigskip\par\noindent
To generate large summation displays, use \TeX\ commands like:
\bigskip\par\noindent\hglue 0.5truein
{\twltt The Maclaurin power series can be written:}
\par\noindent\hglue 0.5truein
{\twltt \$\$ e\^{}$\{$x$\}$ =
$\backslash$sum\_$\{$m=0$\}$\^{}$\{${}%
$\backslash$infty$\}${} $\{$x\^{}$\{$m$\}$ $\backslash$over m!$\}$ =}
\par\noindent\hglue 0.5truein
{\twltt 1+x+%
$\{${}$\{$x\^{}$\{$2$\}${}$\}$ $\backslash$over 2!$\}$ +
$\{${}$\{$x\^{}$\{$3$\}${}$\}$ $\backslash$over 3!$\}$ +
$\backslash$cdots
\$\$}
\par\noindent\hglue 0.5truein
{\twltt $\backslash$medskip$\backslash$par$\backslash$noindent}
\bigskip\par\noindent
That yields output which looks like:
\bigskip\par\noindent\hglue 0.5truein
The Maclaurin power series can be written:
$$ e^{x}=\sum_{m=0}^{\infty} {x^{m} \over m!} =
1+x+{{x^{2}} \over 2!} + {{x^{3}} \over 3!} + \cdots$$
\bigskip\par\noindent
$\bullet$ Summation limits are actually nothing special --- in fact, to
\TeX\ they are simply a subscript and a superscript. (Did you notice
the underscore and caret?)
\bigskip\par\noindent
$\bullet$ Normally, expressions containing summations are large and
complicated and thus end up displayed (i.e., set between 
double dollar signs). If 
you find yourself with summations in embedded (single dollar sign)
equations, you need to be aware that the capital sigma will be smaller
than in displayed form, and the summation limits will be
moved so they follow the summation rather than appearing above and
below the summation. For example, by changing our double dollar signs
to single dollar signs, our Maclaurin example becomes transformed into:
\bigskip\par\noindent
The Maclaurin power series can be written:
$ e^{x}=\sum_{m=0}^{\infty} {x^{m} \over m!} =
1+x+{{x^{2}} \over 2!} + {{x^{3}} \over 3!} + \cdots $
\bigskip\par\noindent
Make a determined effort to set summations in displayed form only.
\bigskip\bigskip\par\noindent
\centerline{\twlbf ^{Integrals}}
\bigskip\bigskip\par\noindent
Integrals are quite similar to summations. Consider the following:
\bigskip\par\noindent\hglue 0.5truein
{\twltt \$\${}$\backslash$int\_$\{$a$\}$\^{}$\{$b$\}$%
f(x)$\backslash$,dx=F(b)-F(a)\$\$}
\bigskip\par\noindent
That comes out looking like:
\bigskip\par\noindent\hglue 0.5truein
$$\int_{a}^{b}f(x)\,dx=F(b)-F(a)$$
\bigskip\par\noindent
Just as in summations, integration limits are simply sub and superscripts;
the {\twltt $\backslash$,} which follows the {\twltt f(x)} 
simply adds a little space between the function and its differential.
\vfill\eject
%%%%%%%%%%%%%%%%%%%%%%%%%%%%%%%%%%%%%%%%%%%%%%%%%%%%%%%%%%%%%%%%%%%%%%%%%%%%%%%
\centerline{\twlbf ^{Definitions}}
^^{Redefining TeX Commands}
^^{Defining TeX Commands}
^^{Command Abbreviations}
\bigskip\par\noindent
By now you may have formed the impression that using
\TeX\ means using impossibly verbose commands to accomplish even the simplest
of tasks.
\bigskip\par\noindent
For example, consider getting
double spaced copy. Right now you get it by saying:
\bigskip\par\noindent\hglue 0.5truein
{\twltt $\backslash$baselineskip=2$\backslash$normalbaselineskip}
\bigskip\par\noindent
Thirty-four characters is a lot to type every time you want
to begin double spacing. 
\bigskip\par\noindent
$\bullet$ Fortunately, \TeX{}'s author allows you to
extend or customize \TeX\ by defining your 
own shorthand commands for frequently employed (but excessively lengthy) 
``native'' \TeX\ instructions. Thus, for instance, we can create a new
\TeX\ command called {\twltt $\backslash$DS} which we'll use as a convenient
way of requesting double spaced copy in \TeX{}:
\bigskip\par\noindent\hglue 0.5truein
{\twltt $\backslash$def%
$\backslash$DS$\{${}$\backslash$baselineskip=2$\backslash$normalbaselineskip%
{}$\}$}
\bigskip\par\noindent
Clearly, it is much easier to type {\twltt $\backslash$DS} when we want 
double spaced copy than it is to type 
{\twltt $\backslash$baselineskip=2$\backslash$normalbaselineskip}, 
don't you think?
\bigskip\par\noindent
Notice that I elected to use a shorthand name for my new command consisting
solely of upper case letters; by doing so, I've practically insured
that my new command won't conflict or interfere with any existing \TeX\ 
commands\footnote{$^{12}$}{Virtually all commands automatically
defined in plain \TeX\ are lowercase or mixed case only, and since \TeX\ 
distinguishes between upper and lowercase letters in parsing command names,
our uppercase-only command names should generally not cause any trouble.}.  
\bigskip\par\noindent
$\bullet$
\TeX\ macros can also take substitutable parameters (or ``macro variables'').
For example, maybe you'd like to have a more general line space setting
macro. In that case, you could define a new command by saying something like:
\bigskip\par\noindent\hglue 0.5truein
{\twltt $\backslash$def$\backslash$SP\#1$\{${}$\backslash$baselineskip=%
\#1$\backslash$normalbaselineskip$\}$}
\bigskip\par\noindent
You could then get triple-spaced copy by saying {\twltt $\backslash$SP 3}
or quad-spaced copy by saying \break {\twltt $\backslash$SP 4}.
\bigskip\par\noindent
$\bullet$ 
If you end up creating a lot of these definitions, collect them all into a
file called {\twltt macro.tex}\ \ You can then automatically include those
definitions in each of your documents by saying:
\bigskip\par\noindent\hglue 0.5truein
{\twltt $\backslash$input macro.tex}
\bigskip\par\noindent
near the top of each of your \TeX\ documents.
\bigskip\par\noindent
$\bullet$ There is a {\twlit lot} more you can do with macros in \TeX{}, 
and a lot more you should {\twlit know}
about \TeX\ macros before you attempt to get too
tricky using them. Aspiring \TeX\ hackers should thoroughly study Chapter
20 of {\twlit The \TeX{}book} for more information before jumping right into
defining a whole pile of macros.
\vfill\eject
%%%%%%%%%%%%%%%%%%%%%%%%%%%%%%%%%%%%%%%%%%%%%%%%%%%%%%%%%%%%%%%%%%%%%%%%%%%%%%%
\centerline{\twlbf V. ^{TEX ON THE OREGON VAX 8800}}
\bigskip\par\noindent
Up to now, we've been talking about \TeX\ in general --- we
haven't really tied ourselves to any particular machine. What we've told you
should work just as well on a PC running \TeX\ as on a NeXT running \TeX\ as 
on a VAX running \TeX{}.
\bigskip\par\noindent
Now, though, it is time to get down to the peculiarities of \TeX{}ing 
on the OREGON VAX. 
\bigskip\par\noindent
\centerline{\twlbf The \TeX\ Execution Cycle} ^^{The TeX Execution Cycle}
\bigskip\par\noindent
The general process you'll be using is as follows:
\bigskip\bigskip\par\noindent
\hglue 1.0truein 
\vbox{\offinterlineskip 
\halign{\hfil #&#&#&#&#&# \hfil\cr
&\vrule&\hrulefill&\vrule&\cr
&\vrule&\strut&\vrule&\cr
&\vrule& \quad \hfill {\twltt \$ @TEX\_SYSTEM:TEX\_LOGIN} \quad\hfill&\vrule&
\qquad&(Define \TeX\ commands for your runs)\cr
&\vrule&\strut&\vrule&\cr
&\vrule&\hrulefill&\vrule&\cr
%%%%%
&&\hfil\strut\hfil&&\cr
&&\hfil\strut\hfil&&\cr
%%%%%
&\vrule&\hrulefill&\vrule&\cr
&\vrule&\strut&\vrule&\cr
&\vrule& \quad \hfill {\twltt \$ ED SAMPLE.TEX} \quad\hfill&\vrule&&
(Make your document)\cr
&\vrule&\strut&\vrule&\cr
&\vrule&\hrulefill&\vrule&\cr
%%%%%
&&\hfil\strut\vrule\hfil&&\cr
&&\hfil\strut\vrule\hfil&&\cr
%%%%%
&\vrule&\hrulefill&\vrule&\cr
&\vrule&\strut&\vrule&\cr
&\vrule& \quad \hfill {\twltt \$ TEX SAMPLE} \quad\hfill&\vrule&&
(\TeX\ it)\cr
&\vrule&\strut&\vrule&\cr
&\vrule&\hrulefill&\vrule&\cr
%%%%%
&&\hfil\strut\vrule\hfil&&\cr
&&\hfil\strut\vrule\hfil&&\cr
%%%%%
&\vrule&\hrulefill&\vrule&\cr
&\vrule&\strut&\vrule&\cr
&\vrule& \quad \hfill {\twltt $<$preview .DVI file$>$} \quad\hfill&\vrule&&
(Not covered in this handout)\cr
&\vrule&\strut&\vrule&\cr
&\vrule&\hrulefill&\vrule&\cr
%%%%%
&&\hfil\strut\vrule\hfil&&\cr
&&\hfil\strut\vrule\hfil&&\cr
%%%%%
&\vrule&\hrulefill&\vrule&\cr
&\vrule&\strut&\vrule&\cr
&\vrule& \quad \hfill {\twltt \$ DVIPS SAMPLE} \quad\hfill&\vrule&&
(Make the {\twltt .DVI} file into PostScript)\cr
&\vrule&\strut&\vrule&\cr
&\vrule&\hrulefill&\vrule&\cr
%%%%%
&&\hfil\strut\vrule\hfil&&\cr
&&\hfil\strut\vrule\hfil&&\cr
%%%%%
&\vrule&\hrulefill&\vrule&\cr
&\vrule&\strut&\vrule&\cr
&\vrule& \quad \hfill {\twltt \$ PSP SAMPLE.PS} \quad\hfill&\vrule&&
(Print it on the laser printer)\cr
&\vrule&\strut&\vrule&\cr
&\vrule&\hrulefill&\vrule&\cr
%%%%%
}}
\bigskip\bigskip\par\noindent
We'll now look at each of those steps, in 
order.\footnote{$^{13}$}{$\ldots$
except for previewing! For information on
^{previewing your document on screen}, 
please see the separate write-up {\twlit
Previewing Your \TeX\ {\twltt .DVI} File On-Screen}}.
\vfill\eject
%%%%%%%%%%%%%%%%%%%%%%%%%%%%%%%%%%%%%%%%%%%%%%%%%%%%%%%%%%%%%%%%%%%%%%%%%%%%%%%
\centerline{\twlbf Building Your \TeX\ Document Using an Editor}
^^{Building Your TeX Document}
^^{Editing Your TeX Document}
\bigskip\par\noindent
$\bullet$ Your first task is to actually build your \TeX\ document.
You can use any of the regular VAX text editors to do this; 
EDT, EVE and EMACS will all work fine, for instance. As an example:
\bigskip\par\noindent
\leftline{\qquad \twltt \$ ED SAMPLE.TEX}
\leftline{\qquad \twltt *C}
\leftline{\qquad \twltt $\backslash$twelvepoint}
\leftline{\qquad \twltt $\backslash$centerline$\{${}$\backslash$bf 
Introduction$\}$}
\leftline{\qquad \twltt $\backslash$bigskip}
\leftline{\qquad \twltt It is well established that the primary 
principle of all Architecture}
\leftline{\qquad \twltt is balance. Without balance, $<$blah blah blah$>$}
\leftline{\qquad \twltt $<$CTRL-Z$>$}
\leftline{\qquad \twltt *EX}
\bigskip\par\noindent
^^{Uploading Your TeX Document From a Micro}
$\bullet$ Some users may prefer to build 
their \TeX\ documents using a microcomputer word
processor or text editor, uploading their file to 
the VAX once it has been prepared. That's
fine, too; just remember to hit return at the end 
of each line of text as you enter it (or save 
your \TeX\ document as ``text with line breaks'').
\bigskip\par\noindent
Upload your \TeX\ document just as you would any other text
file; raw \TeX\ documents consist solely of regular printable 
characters so there is no need to do a ``binary'' file transfer
when uploading a \TeX\ document prepared on a microcomputer.
\bigskip\bigskip\par\noindent
\centerline{\twlbf Defining \TeX}
^^{Defining TeX}
\bigskip\par\noindent
Because a relatively small 
percentage of all VAX users will initially be using \TeX{}, the commands
needed to run \TeX\ are not automatically defined at login time.
Instead, in order to be able to use \TeX{}, you need to enter the command:
\bigskip\par\noindent
\leftline{\qquad \twltt \$ @TEX\_SYSTEM:TEX\_LOGIN}
\bigskip\par\noindent
at the dollar sign. The \TeX\ commands obtained by executing that command
will remain available for the duration of your current session, but will
be lost when you log off.
\bigskip\par\noindent
Therefore, 
if you plan to do a lot of work with \TeX{}, you'll want to add the above
DCL command to your account's {\twltt LOGIN.COM} file so you don't need to 
manually re-request \TeX{}'s commands every time you login to do some
computerized typesetting. (Be sure to actually type in
the dollar sign at the start of this (or any) command you add to 
your {\twltt LOGIN.COM} file!)
^^{LOGIN.COM file}
\vfill\eject
%%%%%%%%%%%%%%%%%%%%%%%%%%%%%%%%%%%%%%%%%%%%%%%%%%%%%%%%%%%%%%%%%%%%%%%%%%%%%%%
\centerline{\twlbf Running \TeX}
^^{Running TeX}
\bigskip\par\noindent
Once you've got your \TeX\ document entered and \TeX{}'s 
commands defined, you're then ready to process your 
document into a device independent ({\twltt .DVI}) 
file by saying:
\bigskip\par\noindent
\leftline{\qquad \twltt \$ TEX filename}
\bigskip\par\noindent
For example, to run \TeX\ on the raw \TeX\ document SAMPLE.TEX, you'd enter:
\bigskip\par\noindent
\leftline{\qquad \twltt \$ TEX SAMPLE}
\bigskip\par\noindent
^^{Default file extension}
Note that you don't have to bother giving the file extension 
of your \TeX\ document file ({\twltt SAMPLE\underbar{.TEX}}) because 
\TeX\ will assume the file name ends in \hbox{\twltt{}.TEX} if you omit the
file extension from the file name. To take advantage of that feature, 
and to help you keep your files organized, you are 
strongly encouraged to have all your \TeX\ document filenames
end in {\twltt .TEX} unless you have a particularly good rationale 
for doing otherwise.
\bigskip\par\noindent
Anyhow, when you run your document through
\TeX\ your screen should look something like:
^^{TeX-ing Your Document}
\bigskip\par\noindent
\leftline{\qquad \twltt \$ TEX SAMPLE}
\leftline{\qquad \twltt This is TeX, NLS VMS Version 2.991a (preloaded format=plain 89.10.26)}
\leftline{\qquad \twltt (DISK\$USER8:[JSMITH]SAMPLE.TEX;3 (TEX\_TEXROOT:[INPUTS]FONTSIZE.TEX;2) [1]}
\leftline{\qquad \twltt [2] [3] [4] [5]} 
\leftline{\qquad \twltt Output written on DISK\$USER8:[JSMITH]SAMPLE.DVI;3 
(5 pages, 9648 bytes).}
\leftline{\qquad \twltt Transcript written on 
DISK\$USER8:[JSMITH]SAMPLE.LIS;3.}
\bigskip\par\noindent
\TeX\ is telling you
that you are running Version 2.991a of the Northlake Software port of 
\TeX\ for VMS, using the normal default \TeX\ format (``{\twltt plain}'')
prepared in October of 1989.
\bigskip\par\noindent
\TeX\ then announces that it is beginning to read and process
a document called {\twltt SAMPLE.TEX}. While processing that file, 
it reads in another file called {\twltt FONTSIZE.TEX} (which we 
know contains \TeX\ commands defining the various fontsizes we might want
to use).
\bigskip\par\noindent 
\TeX\ then begins to build and output pages one at a time: as each page
is output, \TeX\ writes the number of that page inside square brackets
on your terminal.
\bigskip\par\noindent
When \TeX\ finishes processing your document, it provides a summary
explaining how many pages it has created, the size of the 
device independent output file ({\twltt SAMPLE.DVI}), and the 
name of the transcript file ({\twltt SAMPLE.LIS}) 
which contains details about any problems \TeX\ had while processing 
your document.   
\bigskip\par\noindent
If you have encountered no errors, you're now ready to convert the output
{\twltt .DVI} file into PostScript for printing on a laser printer.
\bigskip\par\noindent
In many cases, however, you'll have at least a couple of \TeX\ 
run time errors to resolve. When you encounter errors, see the next 
section for error deciphering assistance. 
\vfill\eject
%%%%%%%%%%%%%%%%%%%%%%%%%%%%%%%%%%%%%%%%%%%%%%%%%%%%%%%%%%%%%%%%%%%%%%%%%%%%%%%
\centerline{\twlbf Decoding \TeX\ Errors}
^^{Decoding TeX Errors}
\bigskip\par\noindent
When \TeX\ has a problem with your document, it will report an error 
message which will probably look roughly like the following:
\bigskip\par\noindent
\leftline{\twltt \$ TEX MYPAPER}
\leftline{\twltt This is TeX, NLS VMS Version 2.991a (preloaded format=plain 89.10.26)}
\leftline{\twltt (DISK\$USER8:[JSMITH]MYPAPER.TEX;2}
\leftline{\twltt ! Undefined control sequence.}
\leftline{\twltt{}l.18 ...small file. Here is some text which 
is $\{${}$\backslash$bg}
\leftline{
\twltt{}~~~~~~~~~~~~~~~~~~~~~~~~~~~~~~~~~~~~~~~~~~~~~~~~~~~~emphasized.$\}$}
\leftline{\twltt ?}
\bigskip\par\noindent
This error message isn't actually too bad, as error messages go. But what
does it mean?
\bigskip\par\noindent
$\bullet$
First of all, \TeX\ is telling you that you are trying to use a 
\TeX\ command which doesn't exist ({\twltt ^{Undefined control sequence}}). 
\bigskip\par\noindent
$\bullet$ Next, \TeX\ tells you the line on which the problem exists 
({\twltt l.18}), and even shows you the point at which the problem was 
detected (\TeX\ marks the point where a problem was detected by dropping 
down to a new line without doing a carriage return). 
\bigskip\par\noindent
In our case, the user entered {\twltt $\backslash$bg} where we can assume
they probably meant to enter {\twltt $\backslash$bf} for ``boldface.'' Fine.
We now know what caused our error. But what do we do
when we're looking at the question mark \TeX\ has issued? 
\bigskip\par\noindent
Your best move as a beginner is to type an {\twltt X} and 
hit return to exit \TeX{}. 
You can then edit your file and fix the line with the problem, and 
rerun \TeX\ on your corrected file.\footnote{$^{14}$}{If you are 
feeling more adventurous, you can type a question mark in response to
\TeX{}'s question mark; \TeX\ will then outline your other 
options for dealing with the reported error.} [Remember that under VMS
you can readily roll back to earlier commands by hitting one or more
$<$CTRL$>$-B's at the dollar sign --- this command ``history'' feature
is very convenient when you repeatedly edit-and-rerun-TEX on a document
during the document development process.
^^{VMS command history feature}
\bigskip\par\noindent
Now we'll explain the most common \TeX\ errors you're likely to encounter. 
\vfill\eject
%%%%%%%%%%%%%%%%%%%%%%%%%%%%%%%%%%%%%%%%%%%%%%%%%%%%%%%%%%%%%%%%%%%%%%%%%%%%%%%
\centerline{\twlbf The Most Common \TeX\ Errors}
\bigskip\par\noindent
^^{Most Common TeX Errors}
In addition to the {\twlbf Undefined control sequence error} we 
just discussed, there are a number of other \TeX\ errors which are also 
commonly encountered. (At least the errors listed below are the sort
of errors {\twlit I} typically see --- {\twlit you} may tend to generate an
entirely different set of errors depending upon your mental processes,
\TeX\ habits, and keyboarding expertise). Anyhow, the errors {\twlit I} 
see most often are:
\bigskip\par\noindent
^^{Overfull hbox}
$\bullet$ {\twlbf Overfull $\backslash$hbox}\quad This error message is
saying that \TeX\ is wrestling with a line of text which is too wide for
the document, and which it cannot break into acceptable width lines (either
by splitting the line at whitespace, or by hyphenating a word near the end of
the line and breaking the line there).
\bigskip\par\noindent
If you get this error, begin by inspecting the text that's making \TeX\ gag. 
Do you have a particularly long and hard-to-hyphenate word near the end of 
the line? \TeX\ may be encountering trouble breaking that 
word.\footnote{$^{15}$}{You may need to give \TeX\ a hand by explaining
where it can safely hyphenate unusual words. Add a {\twltt $\backslash$-}
to tell \TeX\ where it can break a tough word if it needs to do so. 
Don't bother providing this sort of advice unless \TeX\ demonstrates a 
need for it --- \TeX\ normally does a fantastic job of breaking words at 
an appropriate point on its own.}
Have you incorrectly used un-breakable ties ({\twltt \~{}}) instead of 
regular spaces for all or most regular interword spaces? Did you explicitly 
create a too-large unbreakable {\twltt $\backslash$hbox}? 
Did you make a huge table that \TeX\ just can't fit on the page? 
Maybe you've made a monster equation? Can you force a break with 
{\twltt $\backslash$break} somewhere near the end of the line to resolve
this problem? 
\bigskip\par\noindent
If inspection of the raw \TeX\ command file fails to give you insight into
what makes the line appear too long, ignore the {\twltt Overfull 
$\backslash$hbox} message and proceed to {\twltt DVIPS} and {\twltt PSP}
the document, error and all. 
\TeX\ will mark the overfull {\twltt $\backslash$hbox}
with a solid right bar near the right hand side of the page. If after
looking at the printed output you still can't
see what the problem is, come see me and I'll try to help you 
resolve the problem.
\bigskip\par\noindent
^^{Underfull vbox}
$\bullet$ {\twlbf Underfull $\backslash$vbox}\quad This error message usually
arises when \TeX\ has too few lines to fill out a page. Most 
commonly, this is caused by issuing a {\twltt $\backslash$eject} without
a preceding {\twltt $\backslash$vfill} to pad out the rest of the page.
\bigskip\par\noindent
^^{end occurred inside a group at level ...}
$\bullet$ {\twlbf $\backslash$end occurred inside a group at level $<$X$>$}
\quad You missed one or more closing braces somewhere. 
Don't be surprised if pages and pages of your document (instead of just 
the couple of words you may have intended) are typeset in italics, or 
are boldfaced, or are underlined. Go back and insert the missing curly 
braces to eliminate this error.
\bigskip\par\noindent
^^{Too many $\}$'s}
$\bullet$ {\twlbf Too many $\}$'s} \quad Now you've got 
{\twlbf too many} right hand curly braces! The number of right hand curly 
braces should match the number of left hand curly braces. 
\bigskip\par\noindent
You may just need to remove a redundant right hand curly brace;
on the other hand, you may need to snoop around and make sure 
you aren't short a necessary or important left hand curly brace! 
\bigskip\par\noindent
^^{Missing \$ inserted}
$\bullet$ {\twlbf Missing \$ inserted} \quad Usually this means that you've:
(a) attempted to use a math-mode-only symbol (such as 
{\twltt $\backslash$backslash}) while you weren't in math mode, 
(b) you missed a dollar sign at the beginning or end of an equation, or
(c) you entered {\twltt \$} instead of {\twltt $\backslash$\$} when you
actually wanted a dollar sign to appear in your text.
\bigskip\par\noindent
If you want to use a math-mode-only symbol such as 
{\twltt $\backslash$backslash} while entering regular text, 
remember to encapsulate the code for that symbol within an opening and 
a closing dollar sign.
\bigskip\par\noindent
$\bullet$ {\twlbf ^{Missing number, treated as zero}} \quad 
\TeX\ expected a number
(most often as part of a dimension), which you didn't provide. Insert an
appropriate numeric value to eliminate this error.
\bigskip\par\noindent
$\bullet$ {\twlbf ^{Misplaced alignment tab character} \&} \quad 
\TeX\ has detected
an ampersand, which is normally used by \TeX\ as a tab character or as an 
alignment point within a stack of equations, in a context where that didn't
make sense. Were you working on setting up a table or aligning equations? 
\bigskip\par\noindent
If not, you probably just wanted to have an ampersand actually appear in 
your text. Change the \ \ {\twltt \&} \ \ into 
a \ \ {\twltt $\backslash$\&} \ \ to make the ampersand actually show up
in your document.
\bigskip\par\noindent
$\bullet$ {\twlbf ^{Double subscript}} \quad
You have a ``subscript upon a subscript'' (such as {\twltt a\_b\_c})
which is ambiguous. \TeX\ needs to know if you meant:
\medskip\par\noindent\hglue 0.5truein
{\twltt a\_$\{$b\_c$\}$}
\medskip\par\noindent
or
\medskip\par\noindent\hglue 0.5truein
{\twltt $\{$a\_b$\}$\_c}
\medskip\par\noindent
since \TeX\ treats those two cases somewhat differently when typesetting them.
Insert left and right curly braces to clearly show just what is getting
subscripted in your expression.
\bigskip\par\noindent
$\bullet$ {\twlbf ^{Ambiguous; you need another $\{$} and $\}$.} \quad 
This is another error you'll see if you tend to be parsimonious with curly
braces. Most often you'll see this error when you're creating a compound 
fraction and you've entered an ambiguous expression like:
\bigskip\par\noindent\hglue 0.5truein
{\twltt \$ a $\backslash$over b $\backslash$over c \$}
\bigskip\par\noindent
Enter curly braces to clarify ``what you want to have above what''. For
example:
\bigskip\par\noindent\hglue 0.5truein
{\twltt \${}$\{${}a $\backslash$over b$\}$ $\backslash$over c \$}
\bigskip\par\noindent
Remember, however, that ``fractions upon fractions'' are generally a bad
idea, and you should rewrite the expression to eliminate them if at all
possible.
\vfill\eject
%%%%%%%%%%%%%%%%%%%%%%%%%%%%%%%%%%%%%%%%%%%%%%%%%%%%%%%%%%%%%%%%%%%%%%%%%%%%%%%
\centerline{\twlbf ^{Converting Your .DVI File Into PostScript}}
\bigskip\par\noindent
After \TeX\ processes your {\twltt .TEX} file without error, you'll have a
{\twlit device independent}
({\twltt .DVI}) intermediate file which you can then 
process into a final {\twlit device specific}
file for printing on a laser printer.
\bigskip\par\noindent
{\twltt .DVI} files can be processed into device specific output files
suitable for printing on any of a number of different printers. For example,
there are {\twltt .DVI} converters, or ``filters,'' 
for PostScript laser printers,
HP LaserJet printers, certain bit-mapped CRT's, etc.  In this 
write-up we're only going to explain how you can write output for PostScript
printers (such as the Computing Center's ^{Xerox 4045/160}) using the {\twltt
^{DVIPS}} (DVI-to-PostScript) driver written by 
Tomas Rokicki.\footnote{$^{16}$}{If you're using a machine other than
OREGON, you can obtain a copy of {\twltt DVIPS} via anonymous FTP from 
{\twltt NEON.STANFORD.EDU}; look in directory {\twltt pub}.
Users at sites without network access can write Tomas Rokicki, Radical 
Eye Software, Box 2081, Stanford, CA 94039 for information on obtaining a
commercial distribution of DVIPS.}
\bigskip\par\noindent
Essentially, once \TeX\ finally processes your document without complaint,
the procedure for 
converting your {\twltt .DVI} file into PostScript is simple.
All you have to do is say:
\bigskip\par\noindent\hglue 0.5truein
{\twltt \$ DVIPS filename}
\bigskip\par\noindent
For example, if your original \TeX\ document file was named 
{\twltt SAMPLE.TEX}, and \TeX\ processed that file into {\twltt SAMPLE.DVI},
you can then convert it into PostScript by saying:
\bigskip\par\noindent\hglue 0.5truein
{\twltt \$ DVIPS SAMPLE}
\bigskip\par\noindent
After executing that command you should see something which looks like:
\bigskip\par\noindent\hglue 0.5truein
{\twltt \$ DVIPS SAMPLE}
\par\noindent\hglue 0.5truein
{\twltt This is dvips, version 5.392 (C) 1986-90 Radical Eye Software}
\par\noindent\hglue 0.5truein
{\twltt ' TeX output 1990.11.15:1443' -$>$ SAMPLE.PS}
\par\noindent\hglue 0.5truein
{\twltt [tex.pro]. [1] [2] [3] [4] [5]}
\par\noindent\hglue 0.5truein
{\twltt \$}
\bigskip\par\noindent
{\twltt SAMPLE.PS} is the output file containing your document 
in final PostScript form, ready to print on a PostScript laser printer. 
\bigskip\par\noindent
^^{Printing Only Certain Pages of a Document}
$\bullet$ There may be times when you only want to print certain pages of
your document. For instance, you may not have touched the first twenty-four
pages of your document, but you need a copy of all pages from there on. To
get a copy of all pages from 25 through the end of your document, you'd say:
\bigskip\par\noindent\hglue 0.5truein
{\twltt \$ DVIPS -p25 SAMPLE}
\vfill\eject
To get only pages 17 through 28, inclusive, try:
\bigskip\par\noindent\hglue 0.5truein
{\twltt \$ DVIPS -p17 -l28 SAMPLE}
\bigskip\par\noindent
(The second command line argument in the above example begins with the
letter ``el,'' not the number ``one''!)
\bigskip\par\noindent
To get only the first 8 pages of SAMPLE.DVI, use:
\bigskip\par\noindent\hglue 0.5truein
{\twltt \$ DVIPS -n8 SAMPLE}
\bigskip\bigskip\par\noindent
^^{Case of DVIPS command line arguments}
$\bullet$ NOTE: If you are familar with DVIPS from another 
system, you may be surprised
to notice that DVIPS on the VAX automatically lowercases any command 
line arguments you provide. This is a feature (or a bug) caused by compiling
DVIPS using VAX/VMS C, which happily trashes the case of all the command line
arguments it touches. If you need to preserve the case of uppercase (or 
mixed case) command line argument strings, you'll need to enclose those 
command line argument strings inside double quotation marks. 
\bigskip\bigskip\par\noindent
\centerline{\twlbf ^{DVIPS Features}}
\bigskip\par\noindent
$\bullet$ DVIPS (\underbar{\twlbf NOT} \TeX{}) can generate special 
PostScript effects, such as ^{overprint screens}. 
For example, each of the appendices to this
guide was set with a large overprinted Helvetica-Bold letter,
denoting the current appendix. For an 
example of how this was done, look at the \TeX\ source file for
the appendices.
\bigskip\par\noindent
Note that these special effects are strictly a function of {\twltt DVIPS};
it is virtually certain that almost any
other DVI-to-$<$whatever$>$ translator won't be able
to handle generation of these special effects! Think twice before you 
decide you need to have this sort of special effect included in  
your document, since their use will limit its portability.
\bigskip\par\noindent
^^{fonts}
$\bullet$ Another {\twltt DVIPS}-specific feature is use of ^{PostScript fonts} 
instead of, or in addition to, the normal Computer Modern Roman fonts. 
For an example of a document which does this, see
Appendix F. Note that like the use of special effects (such as overprinting),
use of PostScript fonts may limit your ability to use {\twltt DVI}
translators other than {\twltt DVIPS}.
\bigskip\par\noindent
^^{graphics}
$\bullet$ A final {\twltt DVIPS}-specific feature is inclusion of 
^{PostScript graphics} directly as part of your \TeX\ document.
See Appendix G for an example of this.
Be aware that PostScript graphic files tend to be quite large, and use
of this feature may limit the portability of your document to other DVI 
translators. You may also need to hack on the actual PostScript file you
want to include to get it into ``includable'' form. 
\vfill\eject
%%%%%%%%%%%%%%%%%%%%%%%%%%%%%%%%%%%%%%%%%%%%%%%%%%%%%%%%%%%%%%%%%%%%%%%%%%%%%%%
\centerline{\twlbf Printing PostScript Output on the VAX's ^{Xerox 4045/160}}
\bigskip\par\noindent
$\bullet$ After you've run your document through {\twltt TEX} and {\twltt
DVIPS}, you'll have a {\twltt .PS} (PostScript) file which can be 
printed on a PostScript laser printer. 
\bigskip\par\noindent
^^{Printing Your PostScript Output}
^^{PSP}
To print the file {\twltt SAMPLE.PS} on 
the the Xerox 4045/160 PostScript-compatible printer
connected to the OREGON VAX, enter the command:
\bigskip\par\noindent\hglue 0.5truein
{\twltt \$\ PRINT/QUE=SYS\$LASER/SETUP=POSTSCRIPT\ \ SAMPLE.PS}
\bigskip\par\noindent
Your PostScript file will then be queued to print.
\bigskip\bigskip\par\noindent
$\bullet$ Because the above print command is an easy one to fumble, the
Computing Center has defined a special ``PostScript Print'' command that's
a little easier to type. Specifically, instead of having to enter the 
complicated {\twltt PRINT} command shown above to print the file SAMPLE.PS,
you can simply say:
\bigskip\par\noindent\hglue 0.5truein
{\twltt \$\ PSP\ \ SAMPLE.PS}
\bigskip\par\noindent
Why has the Computing Center gone to the trouble of creating a special command
just to prevent possible user problems in entering that {\twltt PRINT} command?
Well, if a user omits either the required {\twltt /QUE} or the required 
{\twltt /SETUP} string, the raw PostScript commands in the user's file will 
be printed as page upon page of seeming gibberish instead of being 
properly interpreted by the printer. This results in a frustrating waste of 
printer resources we'd like to help you avoid if possible.
\bigskip\par\noindent
One small caution about using the PSP command: note that the {\twltt PSP}
command won't automatically pick up any special {\twltt /NOTE} qualifiers 
you may have added to your regular PRINT command (such as those which you 
may have added to route your printer output to a special output box). If you
can't find your PostScript output in your normal printer output box when
you're using the {\twltt PSP} command, be
sure to check out front in the general printer output pick up area, too.
\bigskip\bigskip\par\noindent
$\bullet$ One more point you should be aware of:
be {\bf sure} to explicitly include the {\tt .PS} file extension
when you name the PostScript file you want to print! If you accidentally 
omit the {\tt .PS} file extension, VMS will (incorrectly) assume that you 
actually want to print {\tt filename.{}\underbar{LIS}} instead of 
{\tt filename.{}\underbar{PS}},and you won't get any output whatsoever
(except the print job header page). 
\vfill\eject
%%%%%%%%%%%%%%%%%%%%%%%%%%%%%%%%%%%%%%%%%%%%%%%%%%%%%%%%%%%%%%%%%%%%%%%%%%%%%%%
\centerline{\twlbf VI. CONCLUSION}
\bigskip\par\noindent
\leftline{\twlbf ^{Where from Here?}}
\bigskip\par\noindent
You now know enough about \TeX\ to be able to handle most typesetting projects.
You should be able to prepare typical text-oriented documents, some basic 
tables, and most commonly-seen types of equations. You have become, in effect,
a journeyman \TeX{}nician.
\bigskip\par\noindent
^^{TeX Books}
However, this doesn't mean that you are done learning about \TeX{}. On the 
contrary, now that you understand the basics of using \TeX\ from reading this
write-up, you should feel ready to dig into Knuth's {\twlit \TeX{}book} in
earnest. There are a lot of subtle \TeX\ topics which have gotten cursory 
coverage (at best) in this write-up, but which {\twlit The \TeX{}book} covers
in much greater detail and from a properly grounded context. You {\twlit 
really} want to obtain a copy of the {\twlit \TeX{}book} and read it in 
detail, even going so far as to work each of the little exercises
provided. 
\bigskip\par\noindent
In addition, you may enjoy reading {\twlit \TeX\ For The
Impatient}\footnote{$^{17}$}{Paul W.
Abrahams, et. al., {\twlit \TeX\ For The Impatient}, Addison-Wesley, 
Reading, MA: 1990, 357 pps.}. Like this write-up, {\twlit \TeX\ For the
Impatient} offers a less technical discussion of \TeX\ than Knuth's
{\twlit \TeX{}Book}. 
\bigskip\par\noindent
Also think about picking up a copy of the original write-up
that comes with Tom Rokicki's {\twltt DVIPS} program: it has some interesting
details about DVIPS which we haven't covered in this write-up. Copies
are available at a nominal cost from the Computing Center Documents Room.
\bigskip\bigskip\par\noindent
\leftline{\twlbf ^{What If I Get Stuck?}}
\bigskip\par\noindent
Feel free to come visit if you need help with a \TeX\ problem.
I'll be glad to try to help, and your visit will help me see what topics 
need expanded coverage in the next revision of this write-up.
\bigskip\par\noindent
You may also want to begin following {\twltt ^{comp.text.tex}} in 
{\twltt NEWS}. That newsgroup gives you access to some of the best \TeX\
experts in the country for those times when you are {\twlit really} stuck
and no one can help you locally. However, {\twlbf PLEASE} 
exhaust local resources {\twlbf FIRST} before you post a question to 
the newsgroup --- you really don't want to bother thousands of people
nationwide with a question you could easily get answered locally by asking
me, or by looking in this handout, or by looking in {\twlit The \TeX{}book}.
\bigskip\par\noindent
Another general resource you should be aware of is the \TeX\ Users 
Group, P.O. Box 9506, Providence, Rhode Island 
02940; phone 401-751-7760; FAX 401-751-1071.
When you join ^{TUG} for \$35.00\ (\$25.00 for students) you get a whole 
host of benefits, including a subscription to TUGboat 
(the \TeX\ User Group's
newsletter), information on \TeX\ training, \TeX\ meetings, a 
directory of other TUG members, access to software, etc.
^^{TeX Users Group}
\vfill\eject\end
